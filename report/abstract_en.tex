% !TeX spellcheck = en_US
\likechapternotoc{ABSTRACT}

The thesis: \pageref*{MyLastPage}~p. , \totfig~fig., \tottab~tabl., \total{citenum}~sources and ? appendices.

The theme of this thesis is ``Game theoretic analysis of schedulers in heterogeneous multiprocessor environment''.

The purpose of this work is to construct a simulation system to emulate the process of parallel multiplication of matrices by splitting them into blocks. The second part of the work is the analysis and development of the mathematical model of the process and revaluation of the model with the help of the built simulation system. Alternative methods of solving the problem are also proposed.

Thesis results:
\begin{itemize}
	\item The model for a static scheduler of type minmin is analyzed
	\item Considered the planners minmax, maxmin and maxmax as similar to minmin.
	\item A simulation system was developed to emulate the process of multiplying matrices by block.
	\item other approaches to the problem are considered: numerical methods of optimization, machine learning, etc.
\end{itemize}

The results of this work can be used in the development of a distributed computing system in order to minimize the time of parelle tasks in the conditions of the ability to control the complexity of tasks. Also, based on this work can be analyzed more complex schedulers.

\MakeUppercase{Schedulers, matrix multiplication, game theory, cloud computing.} 