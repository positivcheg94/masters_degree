% !TEX root = ../thesis.tex

\section{Структура симуляційного програмного забезпечення}

Програма написана   на мові C++ та умовно поділяється на дві логічні частини: модуль обробки параметрів та модуль симуляції.

Модуль обробки параметрів дозволяє задавати розмір матриць, режим одноно гравця чи двох, межі перебору стратегій розрізання, параметри планувальника, характеристики обчислювальних модулів, параметрів мережі – bandwidth та latency без перековпіляції програми через параметри командного рядка.

Модуль симуляції генерує задачі для кожного з користувачів, зливає їх в один список та власне подає їх на симулятор, який повертає оброблені задачі з проставленими змінними часу початку та завершення роботи над задачею і номером обчислювального вузла, який обробляю цю  задачу.

Етапи роботи симулятора:
\begin{itemize}
	\item[1.] Перемішування списка очікуючих задач з метою емуляцїї отримання задач у випадковому порядку.
	
	\item[2.] Сортування списка очікуючих задач по складності у відповідності до пріорітетності задач та списка обчислювальних вузлів по потужності у відповідності до пріорітетності обчилювальних вузлів. Наприклад для minmax планувальника список очікуючих задач буде відсортованим від простої до складної, а список обчислювальних вузлів від потужного до повільного.
	
	\item[3.] Ініціалізація початкових задач для обчислювальних вузлів вилучаючи перші елементи з  відсортованих списків очікуючих задач та вузлів, проставлення початкового часу, який на даний момент рівний 0, та обчислення часу очікуваного завершення виконання задачі. Структури з посиланням на задачу, обчислювальний вузол та даними про початок та завершення виконання задачі поміщаються у чергу з пріорітетом по найменшому часу завершення.
	
	\item[4.] Взяти з пріорітетної черги задачу, яка буде найближчою по часу наступною виконаною задачею. Приняти час симуляції за час завершення взятої з черги задачі, додати задачу до списку виконаних задач разом із даними про час її завершення.
	
	\item[5.] Якщо список очікуючих задач не пустий, то вилучити перший елемент, поставити час початку як час симуляції, обчислити час завершення та додати у чергу з прірітетом, поставивши обчислювальний вузол як перший вільний із відсортованого списка вузлів.
	
	\item[6.] Якщо пріорітетна черга не пуста, то повернутися на крок 4.
	
	\item[7.] Знайти у списку виконаних задач найпізніші повернені задачі для кожного з користувачів та повернути їх час завершення.
\end{itemize}
