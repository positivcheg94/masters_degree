\section{Ілюстрація результатів симуляції}

\subsection{Симуляція для одного користувача}
Симуляція для одного користувача в загальному випадку навіть не вважається грою, а більш схоже на звичайну оптимізаційну проблему. Проте графіки симуляцій для одного користувача можуть показати характер обробки задач при блочному розрізанню матриць.

\begin{figure}[H]
	\centering
		\includegraphics[width=\textwidth]{practice/img/one_user_different_proc}
	\caption{Графік залежності часу виконання всіх задач користувача від розміру розрізання для різних кількостей обчислювальних вузлів}
	\label{fig:one_diff_proc}
\end{figure}

На Рис. \ref{fig:one_diff_proc} зображено залежність часу симуляції від розбиття при фіксованих N, latency, bandwidth для 3, 4 та 5 обчислювальних вузлів. Чим більше ОВ, тим швидше множення матриць, проте для деяких розрізань можна побачити майже однаковий час при різній кількості обчислювальних вузлів. Особливо це помітно для 4 та 5, починаючи з розміру розрізання 2500 час для них однаковий хоч для обчислень і задіяно більше ОВ.

\begin{figure}[H]
	\centering
	\includegraphics[width=\textwidth]{practice/img/one_user_different_N}
	\caption{Графік залежності часу виконання всіх задач користувача від розміру розрізання для різних розмірів матриць}
	\label{fig:one_diff_N}
\end{figure}

На Рис. \ref{fig:one_diff_N} показано для розмірів матриць 3000, 4000 та 5000 графіки залежності часу виконання усіх задач користувача від розбиття для 5 ОВ. З нього відносно можна помітити, що графіки мають приблизно однакову форму і можливо між ними має місце звичайна пропрорційну залежність від розміру матриці N.

\subsection{Симуляція для двої користувачів}

\begin{figure}[H]
	\centering
	\includegraphics[width=\textwidth]{practice/img/two_users_fixed_first}
	\caption{Графік залежності часу виконання всіх задач другого користувача від розміру його стратегії розрізання при фіксованих стратегіях першого користувача}
	\label{fig:two_users_fixed_first}
\end{figure}

На Рис. \ref{fig:two_users_fixed_first} зображено залежність часу виконання усіх задач другого користувача від розміру розбиття при фіксованому розбитті користувача 1 для 5 ОВ. На графіку чітко спостерігається стрибки при переході розбиття користувача 2 за фіксоване значення розбиття користувача 1. Це особливість minmin та minmax оскільки вони в першу чергу виконують найлегші задачі, тому користувач, що вибрав менше розбиття, має менший час виконання усіх його задач.


