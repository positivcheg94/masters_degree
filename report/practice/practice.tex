% !TEX root = ../thesis.tex

\chapter{АНАЛІЗ РЕЗУЛЬТАТІВ СИМУЛЯЦІЙ}

% !TEX root = ../thesis.tex

\section{Структура симуляційного програмного забезпечення}

Програма написана   на мові C++ та умовно поділяється на дві логічні частини: модуль обробки параметрів та модуль симуляції.

Модуль обробки параметрів дозволяє задавати розмір матриць, режим одноно гравця чи двох, межі перебору стратегій розрізання, параметри планувальника, характеристики обчислювальних модулів, параметрів мережі – bandwidth та latency без перековпіляції програми через параметри командного рядка.

Модуль симуляції генерує задачі для кожного з користувачів, зливає їх в один список та власне подає їх на симулятор, який повертає оброблені задачі з проставленими змінними часу початку та завершення роботи над задачею і номером обчислювального вузла, який обробляю цю  задачу.

Етапи роботи симулятора:
\begin{itemize}
	\item[1.] Перемішування списка очікуючих задач з метою емуляцїї отримання задач у випадковому порядку.
	
	\item[2.] Сортування списка очікуючих задач по складності у відповідності до пріорітетності задач та списка обчислювальних вузлів по потужності у відповідності до пріорітетності обчилювальних вузлів. Наприклад для minmax планувальника список очікуючих задач буде відсортованим від простої до складної, а список обчислювальних вузлів від потужного до повільного.
	
	\item[3.] Ініціалізація початкових задач для обчислювальних вузлів вилучаючи перші елементи з  відсортованих списків очікуючих задач та вузлів, проставлення початкового часу, який на даний момент рівний 0, та обчислення часу очікуваного завершення виконання задачі. Структури з посиланням на задачу, обчислювальний вузол та даними про початок та завершення виконання задачі поміщаються у чергу з пріорітетом по найменшому часу завершення.
	
	\item[4.] Взяти з пріорітетної черги задачу, яка буде найближчою по часу наступною виконаною задачею. Приняти час симуляції за час завершення взятої з черги задачі, додати задачу до списку виконаних задач разом із даними про час її завершення.
	
	\item[5.] Якщо список очікуючих задач не пустий, то вилучити перший елемент, поставити час початку як час симуляції, обчислити час завершення та додати у чергу з прірітетом, поставивши обчислювальний вузол як перший вільний із відсортованого списка вузлів.
	
	\item[6.] Якщо пріорітетна черга не пуста, то повернутися на крок 4.
	
	\item[7.] Знайти у списку виконаних задач найпізніші повернені задачі для кожного з користувачів та повернути їх час завершення.
\end{itemize}

На Рис. \ref{fig:help_ilustration} зображено основний інтерфейс програми. Програма має деякі обов'язкові параметри та опціональні - тобто такі, для яких вписані значення за вмовчанням, але при потребі їх можна змінити.

\begin{figure}[H]
	\centering
	\includegraphics[width=\textwidth]{practice/img/help_ilustration}
	\caption{Скріншот вікна з описом параметрів програми}
	\label{fig:help_ilustration}
\end{figure}

Список обов'язкових параметрів:
\begin{itemize}
	\item[1.] $problem\_size$ - цей параметр відповідає за визначення розміру матриць, симуляція яких буде проводитись. Розмір визначається одним числом оскільки ми матриці вважаємо квадратними - $N \times N$.
	\item[2.] $mips$ - цей параметр приймає список чисел, які визначають кількість обчислювальних вузлів у розподіленому середовищі та їх потужності. Для простоти було вирішено приймати потужності як коефіцієнти при деякій номінальній потужності, яка задається опціональним параметром $nominal\_mips$. Наприклад параметри "1 1.5 5 6" задають 4 обчислювальних вузла з потужностями $1*nominal\_mips, 1.5*nominal\_mips, 5*nominal\_mips, 6*nominal\_mips$
	\item[3.] $slices$ - параметер відповідає за вибір набору стратегій, симуляція яких буде проводитись. У випадку двох гравців це буде декартовий добуток цієї множини с собою.
\end{itemize}

Список опціональних параметрів:
\begin{itemize}
	\item[1.] $nominal\_mips$ - задає номінальний множник потужностей ОВ. За вмовченням вибраний такий, що відповідає за множення матриць розмірів 2000 за 1.6 секунди, що відносно відповідає потужності одного ядра у сучасних комп'ютерах.
	\item[2.] $bandwidth$ - регулює ширину канала для передачі даних з кожним ОВ.
	\item[3.] $ping$ - виставляє штраф за з'єднання.
	\item[4.] $single$ - булевий параметр $\{0,1\}$, який дозволяє проводити симуляції лише для одного користувача.
	\item[5.] $task\_priority$ - визначає пріорітетність вибору задач. Може приймати 2 значення - $min$ чи $max$. При виборі $min$ спочатку виконуються задачі з меншим обсягом обчислень, а для $max$ навпаки - з більшим. За вмовчуванням береться режим $min$.
	\item[6.] $proc\_priority$ - аналогічно до $task\_priority$ визначае правило вибору вільного процесора. Для $min$ задача планується на вільний процесор з найменшою потужністю, для $max$ з найбільшою.
\end{itemize}



\section{Ілюстрація результатів симуляції}

\subsection{Симуляція для одного користувача}
Симуляція для одного користувача в загальному випадку навіть не вважається грою, а більш схоже на звичайну оптимізаційну проблему. Проте графіки симуляцій для одного користувача можуть показати характер обробки задач при блочному розрізанню матриць.

\begin{figure}[H]
	\centering
		\includegraphics[width=\textwidth]{practice/img/one_user_different_proc}
	\caption{Графік залежності часу виконання всіх задач користувача від розміру розрізання для різних кількостей обчислювальних вузлів}
	\label{fig:one_diff_proc}
\end{figure}

На Рис. \ref{fig:one_diff_proc} зображено залежність часу симуляції від розбиття при фіксованих N, latency, bandwidth для 3, 4 та 5 обчислювальних вузлів. Чим більше ОВ, тим швидше множення матриць, проте для деяких розрізань можна побачити майже однаковий час при різній кількості обчислювальних вузлів. Особливо це помітно для 4 та 5, починаючи з розміру розрізання 2500 час для них однаковий хоч для обчислень і задіяно більше ОВ.

\begin{figure}[H]
	\centering
	\includegraphics[width=\textwidth]{practice/img/one_user_different_N}
	\caption{Графік залежності часу виконання всіх задач користувача від розміру розрізання для різних розмірів матриць}
	\label{fig:one_diff_N}
\end{figure}

На Рис. \ref{fig:one_diff_N} показано для розмірів матриць 3000, 4000 та 5000 графіки залежності часу виконання усіх задач користувача від розбиття для 5 ОВ. З нього відносно можна помітити, що графіки мають приблизно однакову форму і можливо між ними має місце звичайна пропрорційну залежність від розміру матриці N.

\subsection{Симуляція для двої користувачів}

\begin{figure}[H]
	\centering
	\includegraphics[width=\textwidth]{practice/img/two_users_fixed_first}
	\caption{Графік залежності часу виконання всіх задач другого користувача від розміру його стратегії розрізання при фіксованих стратегіях першого користувача}
	\label{fig:two_users_fixed_first}
\end{figure}

На Рис. \ref{fig:two_users_fixed_first} зображено залежність часу виконання усіх задач другого користувача від розміру розбиття при фіксованому розбитті користувача 1 для 5 ОВ. На графіку чітко спостерігається стрибки при переході розбиття користувача 2 за фіксоване значення розбиття користувача 1. Це особливість minmin та minmax оскільки вони в першу чергу виконують найлегші задачі, тому користувач, що вибрав менше розбиття, має менший час виконання усіх його задач.


\begin{figure}[H]
	\centering
	\includegraphics[width=\textwidth]{practice/img/two_users_surface_plot_20_400}
	\caption{Графік залежності часу виконання всіх задач другого користувача для всіх комбінацій стратегій обох користувачів з відрізка $[20, 400]$}
	\label{fig:two_users_surface_plot_20_400}
\end{figure}

На перший погляд поверхя, яка отримана шляхом симуляції усіх можливих пар стратегій обох користувачів з відріка $[20, 400]$, може здааватися гладкою та випуклою Рис. \ref{fig:two_users_surface_plot_20_400}. Проте, пам'ятаючи природу графіків при фіксованій стратегії першого користувача на Рис. \ref{fig:two_users_fixed_first}, слід подивитися на Рис. \ref{fig:two_users_surface_plot_20_400} більш ретельно, наприклад побудувати поверхню усіх комбінацій стратегій з відрізка $[150, 350]$.

\begin{figure}[H]
	\centering
	\begin{subfigure}[b]{0.45\textwidth}
		\includegraphics[width=\textwidth]{practice/img/two_users_surface_plot_150_350_top}
		\caption{Верхня частина графіка}
		\label{fig:two_users_surface_plot_150_350_top}
	\end{subfigure}
	\hfill
	\begin{subfigure}[b]{0.45\textwidth}
		\includegraphics[width=\textwidth]{practice/img/two_users_surface_plot_150_350_bot}
		\caption{Нижня частина графіка}
		\label{fig:two_users_surface_plot_150_350_bot}
	\end{subfigure}
	\caption{Графік залежності часу виконання всіх задач другого користувача для всіх комбінацій стратегій обох користувачів з відрізка $[150, 350]$, \ref{fig:two_users_surface_plot_150_350_top} - фокус на верхню частину поверхні, \ref{fig:two_users_surface_plot_150_350_bot} - фокус на нижню частину поверхні}
	\label{fig:two_users_surface_plot_150_350}
\end{figure}

При кращій деталізації можна чітко побачити, що на Рис. \ref{fig:two_users_surface_plot_150_350_top} спостерігається форма сходинок по всій верхній частині графіка і вона не така проблемна, як нижня частина, показана на Рис. \ref{fig:two_users_surface_plot_150_350_bot}. Оскільки нижня частина більш цікава через те, що час завершення усіх задач користувача там менший, то і глобальний мінімум варто шукати саме на нижній частині. Нижня частина має особливої форми канави і саме вони є основною проблемою.

\begin{table}[H]
		\begin{tabular}{c | c | c | c}
			
			\csvautotabular{practice/csv/5000_min_min_proc3_p0.0_bw1e9.csv}
		
		\end{tabular}
	\caption{Таблиця значень часів повернення усіх задач користувачів для різних стратегій розрізань}
	\label{table:values_table}
\end{table}

З Таблиці \ref{table:values_table} можна побачити як в між деякими сусідніми значеннями спочатку час трохи збільшується, а потім різко зменшується. Таким чином структура функції і проблеми її оптимізації очевидні.









\section*{Висновки до розділу}
\addcontentsline{toc}{section}{Висновки до розділу}

У цьому розділі були проведені експерименти за допомоги побудованої СПЗ та представлені ілюстрації залежностей часу від розбиття для різних конфігурацій обчислювального середовища. Ці експерименти дозволили перевірити основну модель на базі потокової та особливо характер штрафів. Показана проблема використання звичайного методу оптимізації, хоч метод і не застрягав у точках з дуже великим часом.

За допомоги бібліотеки "nashpy" на мові Python було проведено пошук рівноваг за Нешем у грі двох користувачів та результати представлені графічно. Усі рівноваги отримані у мішаних стратегіях, які знаходяться у межах точок з оптимальними часами виконання.