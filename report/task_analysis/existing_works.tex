\section{Аналіз існуючих робіт}

У роботі \cite{AppSelAlgoFEffResProvInCloud} проведений короткий аналіз стратегій виділення ресурсів типів Min-Min та Max-Min. Також у цій роботі занадто проста схема задач, складність обчислень прямо пропорційна розмірам файлів, що буває дуже рідко для трудоемних задач. Ця робота проводить симуляції, в яких кількість задач не більше 10 та всього 2 обчислювальних вузла. Проте вони запропонували алгоритм вибору, який покращує звичайну реалізацію стратегії виділення ресурсів. Також у цій роботі був використаний пакет CloudSim як основний засіб для аналізу та перевірки теорії.

Також цікавою роботою є використання теорії керування з метою мінімізації часу на виконання задачі заданого набору задач \cite{Prasanna1991GeneralisedMS}. У цій роботі використовується time shared політика планування та динамічне переключення між виконанням задач. Для цього використовується теорія оптимального керування, яка керує кількістю часу, яка надається для задачі у наступній ітерації. Також доведена еквівалентність задачі планування і задачі оптимального управління часом.

Проблема ефективного множення великих матриць також розглянута у \cite{PushingTheBoundsOfMatrixMatrix}. У цій роботі розглядають більш стандартизовану операцію над матрицями: $C := A*B + C$. Ця операція є основною у бібліотеках стандарту BLAS (Basic Linear Algebra Subprograms), яка в них називається GEMM (General Matrix Multiply) та описується як $C = \alpha A*B + \beta C$. Робота базується в першу чергу на \cite{IOComplexityMatrixMatrix} та \cite{Irony}, я яких проводиться аналіз складності вводу та виводу у випадку множення матриць і також можливі оптимізації, які можна застосувати у розподіленому середовищі з метою зменшення затрат на операції передачі даних. Ця проблема також випливає і у нашому дослідженні, оскільки показано, що надлишковість передачі даних при розбитті матриць на блоки дуже швидко зростає, проте найбільш оптимальними розбиттями є саме розбиття на малі блоки так як вони мінімізують простоювання вузлів при завершенні виконання блоку задач. Дослідження у роботі виконані саме для випадку блочного розбиття матриць і не підходять для паралельної версії алгоритма Штрассена.

Одна з свіжих робот на тематику ефективного множення матриць це \cite{CodedHeterogeneousMatrixMatrix}. В роботі розглядається операція $y:=Ax$ для матриці $A \in \mathbb{R}^{m*n}$ та вектора $x \in \mathbb{R}^{n}$ як частковий випадок множення двох матриць. Побудована модель обчислень також включає дві схеми: Uncoded Load Balanced (ULB) та Coded Equal Allocation (CEA). Особливістю роботи є оцінка оптимальної конфігуляції обчислювального середовища на базі Amazon AWS завдяки побудові моделі. Знаходиться компроміс між вибором набору кластерів із заплопонованих тарифних планів та швидкістю виконання операцій. Для пошуку оптимальної конфігурації запропонований еврестичний пошук. На основі цієї роботи була запропонована покращена схема кодування та проведені порівняння з існуючими \cite{CodedHeterogeneousMatrixMatrix2}.