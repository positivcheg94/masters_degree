\chapter{АНАЛІЗ ЗАДАЧІ ПЛАНУВАННЯ У ХМАРНОМУ СЕРЕДОВИЩІ}

\section{Планувальники у розподілених обчисленнях}

Планувальники в загальному поділяются на два типи: статичні та динамічні. Статичні мають певне правило розподілу задач на обчислювальні вузли та правило не зміюється під час роботи планувальника. Динамічні планувальники в свою чергу можуть адаптуватися під час роботи та змінювати стратегії планування. Хоч динамічні планувальники і здаються більш універсальними, проте вони дуже складні для аналізу та часто розробляються з метою оптимізації певного параметра.

Основною метою планування є розподіл виконання задач, які надсилають користувачі, між обчислювальними вузлами. Планувальник формує чергу виконання задач, та визначає яку задачу на якому вузлі стартувати. Частіше за все користувачів хвилює лише мінімізація часу завершення деякої відправленої множини задач у хмару.

Структура хмари у загальному випадку описується за допомоги таких сутностей як:
\begin{itemize}
	\item брокер
	\item планувальник
	\item дата центр
	\item хост
	\item віртуальна машина
\end{itemize}

Брокер являєься посередником між користувачем та хмарою і саме він формує запити до хмари на виконання задач та отримує результати їх виконання. Планувальник отримує задачі від брокерів різних користувачів та має свою певну стратегію розподілу черги задач, саме він керує дата центрами. Дата центри це несамостійні сутності і повністю керовані планувальниками. Їх основна мета - надавати обчислювальні вузли, запускати задачі по команді від планувальника та віддавати результати. Також у випадку динамічної хмари по певним запитам вони можуть збільшувати чи зменшувати кількість обчислювальних вузлів у мережі.
Віртуальна машина - це окремий обчислювальний вузол, який вміє лише отримувати вхідні дані, виконувати задачу та відправляти результат у дата центр.

\begin{figure}[H]
	\centering
	\includegraphics[width=\textwidth]{task_analysis/img/cloud_representation}
	\caption{Ілюстрація структури хмарної обчислювальної мережі}
	\label{fig:cloud_representation}
\end{figure}

Існує дві політики планування: space shared та time shared. У політиці space shared задачі у черзі розділяють між собою логічні ядра процесора обчислювального вулза і у випадку якщо усі ядра зайняті, то задач чекає звільнення ядра. Саме ця політика і вікористовується для тестування алгоритмів планування оскільки у випадку коли обчислювальні вузли мають лише одне ядро процесора, задачі виконуються послідовно від початку до кінця. Найпростіший алгоритм, який можна зустріти разом з політикою space shared - first come first served (FCFS).

Етапи роботи space shared політики:
\begin{itemize}
	\item[Крок 1] Формується черга задач
	\item[Крок 2] Запланувати виконання наступної задачі із черги
	\item[Крок 3] При завершенні задачі відправити результат
	\item[Крок 4] Якщо черга непорожня, то повернутися на крок 2
	\item[Крок 5] Кінець
	\item[***] Нові отримані задачі просто додаються у чергу і вони чекають свого виконання
\end{itemize}

Time shared політика в свою чергу означає що задачі у чергі розділяють між собою процесорний час. Тобто не обчислювальні ресурси, а проміжки часу які будуть витрачені на виконання кожної із задач. Усі задачі в черзі стартують в один і той самий час. Задача планувальника в цьому випадку - вирішувати коли потрібно призупинити виконання одної задачі та стартувати чи продовжити виконання іншої задачі. На перший погляд ця політика значно краща оскільки дозволяє виконувати набір задач поступово, проте час переключення від одної задачі до другої також потрібно враховувати. А також щоб стартувати усі задачі потрібно мати вхідні дані для усіх задач, що також не завжди зручно. Разом з цією політикою часто можна зустріти посилання на алгоритм планування Round Robin.

Етапи роботи tune shared політики:
\begin{itemize}
	\item[Крок 1] Формується черга задач.
	\item[Крок 2] Усі задачі запускаються одночасно у режимі переключення між задачами за правилом, яке визначає планувальник
	\item[Крок 3] Кінець
	\item[***] Нові отримані задачі просто додаються у чергу і зразу запускаються та працюють у режимі спільного використання процесорного часу
\end{itemize}

\begin{figure}[H]
	\centering
	\includegraphics[width=\textwidth]{task_analysis/img/time_space_shared}
	\caption{Ілюстрація планування задач для time та space shared }
	\label{fig:time_space_shared}
\end{figure}



\section{CloudSim як засіб для симуляції хмарних обчислень}

Останнім часом технологія хмарних обчислень виникла як провідна технологія для забезпечення надійних, безпечних, відмовостійких, стійких та масштабованих обчислювальних послуг, які представлені як програмне забезпечення, інфраструктура або платформа як послуги (SaaS, IaaS, PaaS). Більше того, ці послуги можуть бути запропоновані в приватних центрах обробки даних (приватні хмари), можуть бути комерційно запропоновані для клієнтів (загальнодоступні хмари), або можливо, що як державні, так і приватні хмари об'єднуються в гібридні хмари.

Ця вже широка екосистема хмарних архітектур, разом із зростаючим попитом на енергоефективні ІТ-технології, вимагає своєчасних, повторюваних та контрольованих методологій для оцінки алгоритмів, програм і політики до фактичного розвитку хмарних продуктів. Оскільки використання реальних тест-сейфів обмежує експерименти до масштабу тест-сейфів і робить відтворення результатів надзвичайно важким завданням, альтернативні підходи для тестування та експериментів сприяють розробці нових технологій Cloud.

Підходящою альтернативою є використання інструментів моделювання, що дає можливість оцінити гіпотезу перед розробкою програмного забезпечення в середовищі, де можна відтворити тести. Зокрема, у випадку обласного обчислення, коли доступ до інфраструктури здійснює платежі в реальній валюті, підходи на основі моделювання дають значні переваги, оскільки це дозволяє Cloud клієнтам протестувати свої послуги в повторимому та контрольованому середовищі безоплатно, а також настроювати продуктивність вузькі місця до розгортання на реальних хмарах. На стороні постачальника, симуляційні середовища дозволяють оцінити різні види сценаріїв лізингу ресурсів при різному розподілі навантаження та ціноутворення. Такі дослідження могли б допомогти постачальникам оптимізувати вартість доступу до ресурсів з упором на підвищення прибутку. За відсутності подібних імітаційних платформ, клієнти Cloud і постачальники повинні спиратися або на теоретичні та неточні оцінки, так і на підходи щодо спроб і помилок, які призводять до неефективної ефективності обслуговування та отримання доходу.

Основна мета цього проекту полягає у забезпеченні узагальненої та розширюваної системи моделювання, яка дозволяє безперешкодно моделювати, моделювати та експериментувати з розвиваються інфраструктурою Cloud Computing та додатковими службами. Використовуючи CloudSim, дослідники та розробники на базі промисловості можуть зосередити увагу на конкретних проблемах дизайну систем, які вони хочуть досліджувати, не занепокоєні деталями низького рівня, пов'язаними з інфраструктурою та службами Cloud.

\subsection{Сильні сторони}

CloudSim фреймворк \cite{CloudSim} достатньо широко охоплює хмарні системи та їх внутрішню структуру, саме 

\subsection{Слабкі сторони}


\section{CloudSim Plus як удосконалена версія CloudSim}

CloudSim Plus \cite{CloudSimPlus} це фреймворк, що є удосконаленою версією CloudSim та все ще розвивається, на відміну від свого батька, який не має оновлень з 2016 року.

\section{Аналіз існуючих робіт}

У роботі \cite{AppSelAlgoFEffResProvInCloud} проведений короткий аналіз стратегій виділення ресурсів типів Min-Min та Max-Min. Також у цій роботі занадто проста схема задач, складність обчислень прямо пропорційна розмірам файлів, що буває дуже рідко для трудоемних задач. Ця робота проводить симуляції, в яких кількість задач не більше 10 та всього 2 обчислювальних вузла. Проте вони запропонували алгоритм вибору, який покращує звичайну реалізацію стратегії виділення ресурсів. Також у цій роботі був використаний пакет CloudSim як основний засіб для аналізу та перевірки теорії.

Також цікавою роботою є використання теорії керування з метою мінімізації часу на виконання задачі заданого набору задач \cite{Prasanna1991GeneralisedMS}. У цій роботі використовується time shared політика планування та динамічне переключення між виконанням задач. Для цього використовується теорія оптимального керування, яка керує кількістю часу, яка надається для задачі у наступній ітерації. Також доведена еквівалентність задачі планування і задачі оптимального управління часом.

Проблема ефективного множення великих матриць також розглянута у \cite{PushingTheBoundsOfMatrixMatrix}. У цій роботі розглядають більш стандартизовану операцію над матрицями: $C := A*B + C$. Ця операція є основною у бібліотеках стандарту BLAS (Basic Linear Algebra Subprograms), яка в них називається GEMM (General Matrix Multiply) та описується як $C = \alpha A*B + \beta C$. Робота базується в першу чергу на \cite{IOComplexityMatrixMatrix} та \cite{Irony}, я яких проводиться аналіз складності вводу та виводу у випадку множення матриць і також можливі оптимізації, які можна застосувати у розподіленому середовищі з метою зменшення затрат на операції передачі даних. Ця проблема також випливає і у нашому дослідженні, оскільки показано, що надлишковість передачі даних при розбитті матриць на блоки дуже швидко зростає, проте найбільш оптимальними розбиттями є саме розбиття на малі блоки так як вони мінімізують простоювання вузлів при завершенні виконання блоку задач. Дослідження у роботі виконані саме для випадку блочного розбиття матриць і не підходять для паралельної версії алгоритма Штрассена.

Одна з свіжих робот на тематику ефективного множення матриць це \cite{CodedHeterogeneousMatrixMatrix}. В роботі розглядається операція $y:=Ax$ для матриці $A \in \mathbb{R}^{m*n}$ та вектора $x \in \mathbb{R}^{n}$ як частковий випадок множення двох матриць. Побудована модель обчислень також включає дві схеми: Uncoded Load Balanced (ULB) та Coded Equal Allocation (CEA). Особливістю роботи є оцінка оптимальної конфігуляції обчислювального середовища на базі Amazon AWS завдяки побудові моделі. Знаходиться компроміс між вибором набору кластерів із заплопонованих тарифних планів та швидкістю виконання операцій. Для пошуку оптимальної конфігурації запропонований еврестичний пошук. На основі цієї роботи була запропонована покращена схема кодування та проведені порівняння з існуючими \cite{CodedHeterogeneousMatrixMatrix2}.



\section*{Висновки до розділу}
\addcontentsline{toc}{section}{Висновки до розділу}

У цьому розділі було розглянуто структуру принципи роботи об'єкта дослідження - хмачного середовища. ХС у наші дні це дуже поширейний інструментр для виконання задач користувачів будь-якої складності та вважається, що воно замінить у певному сенсі домашні обчислювальні машини.

Розглянуті популярні засоби для симуляції хмарного середовища - CloudSim, CloudSim Plus та GridSim. Ці засоби дуже популярні та часто використовуються у бакалаврських чи магістерських роботах для тестування спроектованих планувальників. Ці СПЗ не підходять для даної роботи оскільки вони написані на мові Java та мають низьку швидкодію у випадку великої кількості задач.

Із огляду літератури можна зробити висновнки, що тема актуальна і нові дослідження проводяться навіть для таких простих задач як множення матриць чи добутку матриці на вектор.