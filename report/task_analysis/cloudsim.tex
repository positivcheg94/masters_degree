\section{CloudSim як засіб для симуляції хмарних обчислень}

Останнім часом технологія хмарних обчислень виникла як провідна технологія для забезпечення надійних, безпечних, відмовостійких, стійких та масштабованих обчислювальних послуг, які представлені як програмне забезпечення, інфраструктура або платформа як послуги (SaaS, IaaS, PaaS). Більше того, ці послуги можуть бути запропоновані в приватних центрах обробки даних (приватні хмари), можуть бути комерційно запропоновані для клієнтів (загальнодоступні хмари), або можливо, що як державні, так і приватні хмари об'єднуються в гібридні хмари.

Ця вже широка екосистема хмарних архітектур, разом із зростаючим попитом на енергоефективні ІТ-технології, вимагає своєчасних, повторюваних та контрольованих методологій для оцінки алгоритмів, програм і політики до фактичного розвитку хмарних продуктів. Оскільки використання реальних тест-сейфів обмежує експерименти до масштабу тест-сейфів і робить відтворення результатів надзвичайно важким завданням, альтернативні підходи для тестування та експериментів сприяють розробці нових технологій Cloud.

Підходящою альтернативою є використання інструментів моделювання, що дають можливість оцінити хмарну систему перед розробкою програмного забезпечення в середовищі, де можна відтворити тести. Зокрема, у випадку обласного обчислення, коли доступ до інфраструктури здійснює платежі в реальній валюті, підходи на основі моделювання дають значні переваги, оскільки це дозволяє Cloud клієнтам протестувати свої послуги в контрольованому середовищі безоплатно, а також настроювати продуктивність вузькі місця до розгортання на реальних хмарах. На стороні постачальника, симуляційні середовища дозволяють оцінити різні види сценаріїв лізингу ресурсів при різному розподілі навантаження та ціноутворення. Такі дослідження могли б допомогти постачальникам оптимізувати вартість доступу до ресурсів з упором на підвищення прибутку. За відсутності подібних імітаційних платформ, клієнти Cloud і постачальники повинні спиратися або на теоретичні та неточні оцінки, так і на підходи щодо спроб і помилок, які призводять до неефективної ефективності обслуговування та отримання доходу.

Основна мета цього проекту полягає у забезпеченні узагальненої та розширюваної системи моделювання, яка дозволяє безперешкодно моделювати, моделювати та експериментувати з розвиваються інфраструктурою Cloud Computing та додатковими службами. Використовуючи CloudSim, дослідники та розробники на базі промисловості можуть зосередити увагу на конкретних проблемах дизайну систем, які вони хочуть досліджувати, не занепокоєні деталями низького рівня, пов'язаними з інфраструктурою та службами Cloud.

\subsection{Сильні сторони}

CloudSim фреймворк \cite{CloudSim} достатньо широко охоплює хмарні системи та їх внутрішню структуру. Саме за його допомоги можна перед проектуванням хмарної інфраструктури спочатку провести симуляції та перевірити адекватність спроектованої архітектури мережі.

Пакет працює на базі подій і усі процеси, що моделюються за допомогою CloudSim, реєструються як події. Також як і запит на створення обчислювального вузла, запит на виконання задач та інші. Це дозволяє додавати нові елементи у симуляційну систему без змін в існуючих файлах, потрібно лише створити свою власну подію та додати її обробку до сутностей, в яких ця подія відбувається. Додавання обробки події часто виконується через наслідування від основного об'єкта та імплементації метода "processEvent".

Така система дуже гнучка та дозволяє швидко написати свою власну симуляцію.

\subsection{Слабкі сторони}

СПЗ CloudSim написане на мові програмування Java та непридатне для моделювання ситуацій з великою кількістю задач. Це одна із проблем, яка і привела до написання власної спрощеної версії СПЗ на мові C++, яка скоротила час симуляцій майже в 50 разів у порівнянні з модифікованою версією СПЗ, яка була написана на базі CloudSim та була майже у 150 разів швидшою за звичайну версію.

Також сама по собі мова Java неадекватно працює у випадку коли потрібно швидко створити велику кількість малих об'єктів, провести з ними певні операції та видалити.

Наприклад, для симуляції множення двох матриць $N*N$ та $N*N$, при розмірі розрізання $n=1$ для одного користувача буде створено $N*N$ задач множення підматриць розмірів $1*N$ та $N*1$. Звісно цей приклад не має сенсу оскільки розрізання $n=1$ скоріше за все не ефективне, проте сама неможливість його змоделювати для $N=5000$ вважається великим недоліком, оскільки поставивши $N=50000$ та $n=10$ отримаємо таку ж саму кількість задач, і лише на їх створення на мові Java витрачається більше 5 секунд. У той час як симуляція, написана на мові С++ дозволяє за ці 5 секунд провести 2 симуляції з такими ж параметрами.

Також слід зазначити, що першою спробою написання симуляції для цієї роботи було саме використання CloudSim. Проте під час розробки програми доводилось дуже довго вивчати документацію і виправляти деякі внутрішні недоліки. Наприклад, у системі була знайдена проблема, що велика кількість малих задач оброблялась повністю і обривалась у випадковому місці. Це пов'язано з тим, що система не розроблялась з метою проведення важких симуляцій.

Також під час проведення симуляцій на виправленій версії системи було помічено дуже повільну швидкість симуляцій для великої кількості задач. Було знайдено 2 основні точки, які сильно сповільнювали програму та виправлені. Одна із правок дала пришвидшення симуляцій приблизно у 50 разів, а друга ще у приблизно 100 разів. Проте навіть з цими правками симуляції проходять значно повільніше за написаний нами симуляційний пакет.

\subsection{CloudSim Plus}

CloudSim Plus \cite{CloudSimPlus} це фреймворк, що є удосконаленою версією CloudSim та все ще розвивається, на відміну від свого батька, який не має оновлень з 2016 року. Оскільки пакет розробляється великою кількістю людей у їх вільний час, то довіра до нього сильно падає. Немає ніякого контролю якості системи, особливо ігнорується швидкодія. Також цей пакет може містити помилки у коді, які виправити сторонньому користувачу буде дуже важко, оскільки для цього потрібно знати та розуміти архітектуру саме цього пакета.

Проте саме цей пакет дозволяє проводити симуляції паралельно і на комп'ютерах з декількома ядрами можна отримати пришвидшення набору симуляцій за рахунок паралельного їх запуску. Саме це і було зроблено у першій версії симуляційного пакету. Але швидкодії все одно було недостатньо, оскільки бажаним результатом було отримати симуляційну систему, яка швидко зможе проводити симуляції усіх можливих комбінацій стратегій двох гравців. І навіть для розміру матриць 1000 симуляція усіх комбінацій проходила більше години