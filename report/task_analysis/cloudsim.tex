\section{CloudSim як засіб для симуляції хмарних обчислень}

Останнім часом технологія хмарних обчислень виникла як провідна технологія для забезпечення надійних, безпечних, відмовостійких, стійких та масштабованих обчислювальних послуг, які представлені як програмне забезпечення, інфраструктура або платформа як послуги (SaaS, IaaS, PaaS). Більше того, ці послуги можуть бути запропоновані в приватних центрах обробки даних (приватні хмари), можуть бути комерційно запропоновані для клієнтів (загальнодоступні хмари), або можливо, що як державні, так і приватні хмари об'єднуються в гібридні хмари.

Ця вже широка екосистема хмарних архітектур, разом із зростаючим попитом на енергоефективні ІТ-технології, вимагає своєчасних, повторюваних та контрольованих методологій для оцінки алгоритмів, програм і політики до фактичного розвитку хмарних продуктів. Оскільки використання реальних тест-сейфів обмежує експерименти до масштабу тест-сейфів і робить відтворення результатів надзвичайно важким завданням, альтернативні підходи для тестування та експериментів сприяють розробці нових технологій Cloud.

Підходящою альтернативою є використання інструментів моделювання, що дає можливість оцінити гіпотезу перед розробкою програмного забезпечення в середовищі, де можна відтворити тести. Зокрема, у випадку обласного обчислення, коли доступ до інфраструктури здійснює платежі в реальній валюті, підходи на основі моделювання дають значні переваги, оскільки це дозволяє Cloud клієнтам протестувати свої послуги в повторимому та контрольованому середовищі безоплатно, а також настроювати продуктивність вузькі місця до розгортання на реальних хмарах. На стороні постачальника, симуляційні середовища дозволяють оцінити різні види сценаріїв лізингу ресурсів при різному розподілі навантаження та ціноутворення. Такі дослідження могли б допомогти постачальникам оптимізувати вартість доступу до ресурсів з упором на підвищення прибутку. За відсутності подібних імітаційних платформ, клієнти Cloud і постачальники повинні спиратися або на теоретичні та неточні оцінки, так і на підходи щодо спроб і помилок, які призводять до неефективної ефективності обслуговування та отримання доходу.

Основна мета цього проекту полягає у забезпеченні узагальненої та розширюваної системи моделювання, яка дозволяє безперешкодно моделювати, моделювати та експериментувати з розвиваються інфраструктурою Cloud Computing та додатковими службами. Використовуючи CloudSim, дослідники та розробники на базі промисловості можуть зосередити увагу на конкретних проблемах дизайну систем, які вони хочуть досліджувати, не занепокоєні деталями низького рівня, пов'язаними з інфраструктурою та службами Cloud.

\subsection{Сильні сторони}

CloudSim фреймворк \cite{CloudSim} достатньо широко охоплює хмарні системи та їх внутрішню структуру, саме 

\subsection{Слабкі сторони}


\section{CloudSim Plus як удосконалена версія CloudSim}

CloudSim Plus \cite{CloudSimPlus} це фреймворк, що є удосконаленою версією CloudSim та все ще розвивається, на відміну від свого батька, який не має оновлень з 2016 року.