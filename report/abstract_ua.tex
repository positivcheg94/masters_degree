% !TeX spellcheck = uk_UA
\likechapternotoc{РЕФЕРАТ}

Дипломна робота: \pageref*{MyLastPage}~ст. , \totfig~рис.,  \tottab~табл., \total{citenum}~джерел та ? додатки.

Метою данної роботи є побудувати симуляційну систему для емуляції процесу паралельного множення матриць шляхом розбиття їх на блоки. Другою частиною роботи є аналіз та розробка математичної моделі процесу та переверки моделі за допомоги побудованої симуляційної системи. Також запропоновані альтернативні методи вирішення проблеми.

Результати роботи:
\begin{itemize}
	\item проаналізована модель для статичного планувальника типу minmin
	\item розглянуті планувальники minmax, maxmin та maxmax як подібні до minmin.
	\item розроблена симуляційна система для емуляції процесу множення матриць поблочно.
	\item розглянуті інші підходи до вирішення проблеми: числові методи оптимізації, машинне навчання тощо.
\end{itemize}

Результати данної роботи можна використати при розробці системи розподілених обчислень з метою мінімізації часу виконання парелельних задач в умовах можливості регулювання складностей задач. Також на базі цієї роботи можна проаналізувати більш складні планувальники.

%Ключові слова:
\MakeUppercase{Планувальники, множення матриць, теорія ігор, хмарні обчислення.}