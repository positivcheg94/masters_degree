% !TeX spellcheck = uk_UA
\likechapternotoc{РЕФЕРАТ}

Магістерська дисертація: \pageref*{MyLastPage}~ст. , \totfig~рис.,  \tottab~табл., \total{citenum}~джерел та ? додатки.

Темою роботи є "Теоретико-ігровий аналіз планувальників у гетерогенному багатопроцесорному середовищі".

Робота актуальна оскільки проблема ефективних обчислень існує та чимало досліджень проводиться у цьому напрямку із застосуванням різних підходів, у тому числі і теорії ігор.

Метою дослідження є пошук рівновах та рішень гри множення матриць у розподіленому середовищі з двома користувачами. Спочатку потрібно побудувати модель множення матриці і для її перевірки розробити симуляційну систему, яка надасть можливість проводити експерименти значно швидше ніж у реальному хмарному середовищі. Розроблена система повинна мати достатню швидкодію для симуляції усіх комбінацій стратегій двох гравців за короткий час.

Об'єктом дослідження є планувальники типу extr-extr у розподіленому середовищі. Предмет дослідження - пошук рівноваг та інших оптимальних точок у грі одного та двох гравців.

Дослідження проводиться методом наукового моделювання процесу блочного множення матриці. За основу взята потокова модель, яка розглянута для неперервного випадку, і у подальшому звужена до дискретної моделі. На другому етапі дослідження проводяться експерименти за допомоги розробленої симуляційної системи, які дозволяють оцінити точність побудованої математичної моделі та дослідити гру на наявність рівноваг.

Новизна результатів полягає у тому, що розроблена симуляційна система, яка дозволяє проводити експерименти та досліджувати хмарне середовище за значно коротший час на простому комп'ютері. Також ця система дозволяє дослідити окремо кожну з компонент функції загального часу. Ще одною особливістю можна вважати універсальність, завдяки якій можна розширювати можливості симуляційної системи і таким чином моделювати більший набір задач.

Результати роботи:
\begin{itemize}
	\item проведено аналіз планувальників extr-extr;
	\item розроблена симуляційна система для емуляції процесу множення матриць поблочно;
	\item розглянуті альтернативні підходи до вирішення проблеми: числові методи оптимізації, машинне навчання тощо.
\end{itemize}

Результати данної роботи можна використати при розробці системи розподілених обчислень з метою мінімізації часу виконання парелельних задач в умовах можливості регулювання складностей задач при розбитті їх на підзадачі. Подальші дослідження можуть бути проведені у напрямі аналізу стандартних операцій лінійної алгебри стантарта BLAS.

\MakeUppercase{планувальники, множення матриць, теорія ігор, хмарні обчислення, рівновага.}