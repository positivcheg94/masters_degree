\chapter{Керування стартапом проекту}
\section{Опис ідеї проекту}
Назва проекту - ``Efficient task distribution for cloud computing". Проект являє собою систему, яка дозволяє правильно розбивати математичны задачі на підзадачі з метою мінімізації часу їх виконання у розподіленому середовищі.

Проект вирішує проблему виконання складних задач, які можна розділити на велику кількість простіших задач, у розподіленому середовищі з мінімізацією часу виконання задач для багатьох користувачів. В першу чергу така проблема може виникнути у дослідницьких центрах, де часто виконуються прості операції проте з неймовірно виликими обсягами даних. Часто такі задачі дуже легко розбивати на більш прості підзадачі, проте не завжди просто оцінити найефективнішу стратегію розбиття. На даний момент такі задачі вирішуються звичайним паралелізмом і такий метод відносно задовольняє потреби, проте основною метою проекта є пошук найефективніших шляхів паралельного виконання дрібних задач та використання симуляційних систем для аналізу ефективності розбиття задачі на підзадачі. 

\begin{table}[H]
	\begin{center}
		\begin{tabular}{|p{5.5cm}|p{5.5cm}|p{5.5cm}|}
			\hline
			\bf{Зміст ідеї} & \bf{Напрямки застосування} & \bf{Вигоди для користувача}
			\\ \hline
			\multirow{2}{6cm}{
				Розробка симуляційної системи, яка допоможе симюлювати обчислення у Cloud системі за значно коротший час без застостування додаткових обчюслювальних машин.
			}                           
			& 1. У сфері досліджень планувальників, перевірка теорій, аналіз особливостей планувальників
			& Можливість провести симуляцію роботи cloud системи із застосуванням вибраного планувалника без значних затрат часу чи грошей
			\\ \cline{2-3}
			
			& 2. Проведення досліждень задач та знаходження найефективніших розбиттів.
			& Симуляція роботи cloud системи при великих обсягах задач за значно коротший час
			\\ \hline
		\end{tabular}
	\end{center}
	\caption{Опис ідеї стартап-проекту}
\end{table}

\begin{table}[H]
	\begin{center}
		\begin{tabular}{|p{0.5cm}|p{2cm}|p{1.5cm}|p{1.5cm}|p{1.5cm}|p{1.5cm}|p{1.5cm}|p{2cm}|p{2cm}|}
			\hline
			
			\multirow{2}{0.5cm} { \bf{№ п/п}  }
			
			& \multirow{2}{2cm} { \centering \bf{Техніко-економічні характеристики ідеї} }
			
			& \multicolumn{4}{p{6cm}|}{ \centering \bf{(потенційні) товари/концепції конкурентів} }
			& \multirow{2}{1.5cm}{ \centering \bf{W (слабка сторона)} }
			& \multirow{2}{1.5cm}{ \centering \bf{N (нейтральна сторона)} }
			& \multirow{2}{1.5cm}{ \centering \bf{S (сильна сторона) } }
			\\[2cm] \cline{3-6}
			& & Мій проект & Cloud Sim & Grid Sim & Amazon AWS & & &
			\\ \hline
			
			1 & Швидка симуляційна система & продукт на мові C++ & бібліо- тека на Java & бібліо- тека на Java & повно- маштабна система обчислень & потребує ретельного аналізу cloud систем & доступна платформа для наукових досліджень планувальників & дозволяє базово оцінити ефективність cloud системи
			\\ \hline
			
			2 & Ефективна аналітика паралельних алгоритмів & додат- кові функції для пошуку оптимальних конфігурацій алгоритма & - & - & розроб- ка власного пакету аналітики & потре- бує значних витрат у дослідження & - & дозволяє значно прискорити обчислення, що у майбутньому дозволить економити на обчисленнях у cloud системах
			\\ \hline
		\end{tabular}
	\end{center}
	\caption{Визначення сильних, слабких та нейтральних характеристик ідеї проекту}
\end{table}


\begin{table}[H]
	\begin{tabular}
		{|l|p{3.5cm}|p{3.5cm}|p{3.5cm}|p{3.5cm}|} \hline
		
		\bf{№ п/п}
		& \bf{Ідея проекту}
		& \bf{Технології її реалізації}
		& \bf{Наявність технологій}
		& \bf{Доступність технологій}
		\\ \hline
		
		1
		& Побудова симуляційної системи для тестування cloud систем
		& На мові C++ реалізація системи, що аналогів по швидкодії для якої немає
		& Технологію потрібно доробити, сама ідея протестована і показує непогані результати по швидкодії, потрібно лише розширити її можливості
		& Технології доступні усім користувачам
		\\ \hline
		
		2
		& Аналіз швидкодії паралельних алгоритмів у cloud системах
		& Залучення команди математиків до аналізу найпопулярніших планувальників та побудови математичних моделей основних задач з лінійної алгебри
		& Не наявні, потрібно запускати процес з нуля
		& Доступні
		\\ \hline
	\end{tabular}
	\caption{Технологічна здійсненність ідеї проекту}
\end{table}

\section{Аналіз ринкових можливостей запуску стартап-проекту}

\begin{table}[H]
	\begin{center}
		\begin{tabular}
			{|l|p{6cm}|p{4cm}|}\hline
			\bf{№} & \bf{Показники стану ринку(найменування)} & \bf{Характеристика} \\ \hline
			1 & Кількість головних гравців, од. & 3 \\ \hline
			2 & Загальний обсяг гравців, \$ & 4.5 млрд. \\ \hline
			3 & Динаміка ринку & Попит зростає, пропозиція майже не збільшується \\ \hline
			4 & Обмеження для входу на ринок& Наявність якісного програмного продукту \\ \hline
			5 & Специфічні вимоги до стандартизації та сертифікації& Відсутні \\ \hline
			6 & Середня норма рентабельності в галузі & 100\% \\ \hline
		\end{tabular}
	\end{center}
	\caption{Попередня характеристика потенційного ринку стартап-проекту}
\end{table}

\begin{table}[H]
	\begin{tabular}
		{|l|p{3.5cm}|p{3.5cm}|p{3.5cm}|p{3.5cm}|} \hline
		\bf{№ п/п} & \bf{Потреба, що формує ринок} & \bf{Цільова аудиторія(цільові сегменти ринку)} & 
		\bf{Відмінності у поведінці різних потенційних цільових груп} & \bf{Вимоги користувачів до товару} \\ \hline
		
		1
		& Необхідність аналізу різних планувальників з метою побудови та перевірки математичних моделей
		& Дослідницькі центри та університети
		& У різних центрах чи університетах проводяться різні дослідження і 
		& Простота у використанні та достатня швидкодія для перевірки гіпотез
		\\ \hline
		
		2
		& Симуляція ресурсоємних задач з розбиттям їх на підзадачі
		& Університети, дослідницькі центри та компанії, які хочуть збільшити ефективність обчислень
		& Кожна окрема задача потребує особливий аналіз та алгоритм розбиття
		& Симуляційна система повинна бути достатньо універсальною оскільки задачі можуть бути не лише чисто математичні
		\\ \hline
	\end{tabular}
	\caption{Характеристика потенційних клієнтів стартап-проекту}
\end{table}

\begin{table}[H]
	\begin{tabular}
		{|l|p{4cm}|p{4cm}|p{4cm}|} \hline
		№ п/п
		& Фактор
		& Зміст загрози
		& Можлива реакція компанії
		\\ \hline
		
		1
		& Платформи для хмарних обчислень розробляють функцію симуляції обчислень за значно меншою ціною
		& Поява функції симуляції на платформах хмарних обчислень за малою ціною може перетягнути значну частину клієнтів на сторону платформ
		& Перегляд цін на підписки, покращення співпраці з обчислювальними центрами та надання додаткових послуг по аналізу моделей задач
		\\ \hline
		
		2
		& Поява продукту з аналогічною швидкодією та відкритим кодом
		& Приводить до повного знецінення товару оскільки відкритий код означає, що продукт доступний усім безкоштовно
		& Розробка додаткових функцій та графічного інтерфейсу для полегшення процесу роботи з програмним продуктом
		\\ \hline
	\end{tabular}
	\caption{Фактори загроз}
\end{table}

\begin{table}[H]
	\begin{tabular}
		{|l|p{4cm}|p{4cm}|p{4cm}|} \hline
		№ п/п
		& Фактор
		& Зміст можливості
		& Можлива реакція компанії
		\\ \hline
		
		1
		& Співробітництво з платформами хмарних обчислень
		& Можливість просування продукта напряму на платформах завдяки угоді з компаніями провайдерами хмарних обчислень
		& Значні збільшення продажів підписок та простіша реклама
		\\ \hline
		
		2
		& Збільшення попиту на складні обчислення
		& Можливість швидкого росту завдяки аналізу популярних високонавантажених алгоритмів
		& Збільшення продажів
		\\ \hline
	\end{tabular}
	\caption{Фактори можливостей}
\end{table}


\begin{table}[H]
	\begin{tabular}
		{|l|p{4cm}|p{4cm}|p{4cm}|} \hline
		№ п/п
		& Особливості конкуретного середовища
		& В чоему проявляється дана характеристика
		& Вплив на діяльність підприємства
		\\ \hline
		
		1
		& Олігополія
		& Існують відриті системи - CloudSim та GridSim
		& Задають стантарт програмного забезпечення для симуляцій хмарних обчислень
		\\ \hline
		
		2
		& Міжнародний рівень конкуретної боротьби
		& Цифрові продукту інформаційного пошуку не мають кородонів
		& Універсальність продукту та швидкодія
		\\ \hline
		
		3
		& Галузева конкуренція
		& Конкуренція проходить у галузі аналізу паралельних алгоритмів
		& Надавати високоефективний продукт для аналізу складних задач
		\\ \hline
		
		4
		& Товарно-видова конкуренція за видами товарів
		& Наявність функцій для повноцінної симуляцій Cloud системи
		& Спостереження за Cloud системами да додавання нових функцій
		\\ \hline
		
		5
		& За характером конкурентних перваг: нецінова
		& В першу чергу важливі швидкодія та універсальність
		& Розширення функціоналу, додавання нових моделей у продукт
		\\ \hline
		
		6
		& За інтенсивністю - марочна конкуренція
		& На ринку я аналогічні продукти зі схожим функціоналом, проте їх швидкодія недостатня для серйозних досліджень
		& Додавання нового функціоналу у продукт
		\\ \hline
		
	\end{tabular}
	\caption{Ступеневий аналіз конкуренції на ринку}
\end{table}

\begin{table}[H]
	\begin{tabular}
		{|l|p{10cm}|} \hline
		\bf{Назва характеристики} & \bf{Характеристика}
		\\ \hline
		
		Прямі конкуренти & Cloudsim Plus
		\\ \hline
		
		Потенційні конкуренти & Goolgle, Microsoft, Amazon AWS, Digital Ocean
		\\ \hline
		
		Постачальники & Amazon, Digital Ocean, Google
		\\ \hline
		
		Клієнти & KPI, Amazon, інші університети
		\\ \hline
		
		Товари-замінники & CloudSim, GridSim, CloudSim Plus
		\\ \hline
	\end{tabular}
	\caption{Аналіз конкуренції за М. Портером (Складові аналізу)}
	\begin{tabular}
		{|l|p{10cm}|} \hline
		\bf{Назва характеристики} & \bf{Характеристика}
		\\ \hline
		
		Прямі конкуренти & Прямі конкуренти - дрібні компанії
		\\ \hline
		
		Потенційні конкуренти & Потенційні конкуренти - компанії-гіганти
		\\ \hline
		
		Постачальники & Постачальники хмарних обчислень - потенційні конкуренти
		\\ \hline
		
		Клієнти & Для клієнтів продуктивність паралельного алгоритма - основна проблема
		\\ \hline
		
		Товари-замінники & Товари-замінники працюють дуже повільно та непридатні для складних симуляцій
		\\ \hline
	\end{tabular}
	\caption{Аналіз конкуренції за М. Портером (Висновки)}
	
\end{table}

\begin{table}[H]
	\begin{tabular}
		{|l|p{7cm}|p{7cm}|} \hline
		№ п/п
		& Фактор конкурентноспроможності
		& Обґрунтування (чинники, що роблять фактор для порівння конкурентних проектів значущим)
		\\ \hline
		
		1
		& Швидкодія
		& Швидкодія значно більша ніж у аналогів написаних на мові Java
		\\ \hline
		
		2
		& Готова база простих планувальників
		& Дає можливість використовувати та досліджувати основні планувальники без велеких зусиль
		\\ \hline
		
		3
		& Простота використання
		& У конкурентів програмні продукти часто дуже складні та потребують багато часу на аналіз різних прикладів перед самою імплементацією симуляції. Також через складну структуру продукту іноді в коді з'являються помилки, які складно виявити на перший погляд.
		\\ \hline
		
		4
		& Універсальність
		& Дозволяє моделювати структури Cloud середовищ будь-якої складності
		
		\\ \hline
	\end{tabular}
	\caption{Обґрунтування факторів конкурентноспроможності}
\end{table}


\begin{table}[H]
	\centering
	\begin{tabular}{|c|p{8cm}|p{1cm}|c|c|c|c|c|c|c|}
		\hline
		\multirow{2}{*}{\#} & \multirow{2}{*}{Фактор конкурентноспроможності}                 & \multirowcell{2}{Бали\\ 1-20} & \multicolumn{7}{c|}{Рейтинг відносно Cloudsim Plus} \\ \cline{4-10}
		
							&                                              					  &                            &  -3 & -2  & -1  & 0 & 1 & 2 & 3 \\ \hline
		1                   & Швидкодія								            			  & 20                         &     &     &     &   &   &   & + \\ \hline
		2                   & Готова база простих планувальників                              & 20                         &     &     &     &   &   &   & + \\ \hline
		3                   & Простота використання           								  & 15                         &     &     &     &   &   & + &   \\ \hline
		3                   & Універсальність										          & 15                         &     &  +  &     &   &   &   &   \\ \hline
	\end{tabular}
	\caption{Порівняльний аналіз сильних та слабких сторін}
\end{table}

\begin{table}[H]
	\centering
	\begin{tabular}{|p{5cm}|p{5cm}|}
		\hline
		Сильні сторони: & Слабкі сторони:
		\\
		
		\tabitem Висока швидкодія у порівнянні з аналогічними продуктами & \tabitem Недостатня універсальність оскільки продукт лише у початковому виді
		\\
		\tabitem Вбудовані прості планувальники, які легко використовувати & \tabitem Продукт постачається мовою C++, яка вважається складнішою за Java для людей, які більше  математики ніж програмісти
		\\
		\tabitem Простота бібліотеки & 
		\\ \hline
		
		Можливості: & Загрози:
		\\
		\tabitem Інтеграція у реальні Cloud системи з метою попереднього аналізу задачі перед саме замовленням хмари & \tabitem Компанії гіганти можуть випустити власне програмне забезпечення для попереднії симуляцій
		\\
		\tabitem Проводити дослідження планувальників & \tabitem Існуючі продукти у розробці також можуть активізуватися та спробувати конкурувати на ринці
		\\
		\tabitem Просте тестування математичних моделей оскільки симуляції достатньо швидкі & \tabitem Поява нових конкурентів з поріняно схожою швидкодією програмного продукту
		\\ \hline
		
	\end{tabular}
	\caption{SWOT- аналіз стартап-проекту}
\end{table}

\begin{table}[H]
	\centering
	\begin{tabular}{|c|p{4cm}|p{4cm}|p{4cm}|} \hline
		№ п/п
		& Альтернатива (орієнтований комплекс заходів) ринкової поведінки
		& Ймовірність отримання ресусрів & Строки реалізації
		\\ \hline
		
		1
		& Розробка API для інших мов програмування з метою полегшення процесу використання програмного продукту
		& 0.9
		& 0.5 року
		\\ \hline
		
		2
		& Додавання специфічних планувальників у симуляційну систему
		& 0.4
		& 0.5 року
		\\ \hline
		
		3
		& Побудова математичних моделей популярних задач лінійної алгебри
		& 0.4
		& 2 роки
		\\ \hline
	\end{tabular}
	\caption{Альтернативи ринкового впровадження стартап-проекту}
\end{table}

\section{Розробка ринкової стратегії проекту}

\begin{table}[H]
	\begin{tabular}
		{|p{0.5cm}|p{2.8cm}|p{3cm}|p{3cm}|p{3cm}|p{3cm}|}
		\hline
		№ п/п
		& Опис профілю цільової групи потенційних клієнтів
		& Готовність споживачів сприйти продукт
		& Орієнтовний помит цільової групи (сегменти)
		& Інтенсивність конкуренції в сегменті
		& Простота входу у сегмент
		\\ \hline
		
		1
		& Університети та дослідницькі центри
		& Потребують простого інтерфейсу та високої швидкодії. Оскільки ці складові наявні, то споживачі будут задоволені
		& Університети потребують продукт для швидкої перевірки теорій. Дослідницькі центри - прискорити власні обчислення
		& Конкуренція із відкритими безкоштовними рішеннями: CloudSim, CloudSim Plus, GridSim
		& Оскільки існуючі рішення дуже повільні для випадків симуляції складних паралельних задач, то вхід у сегмент дуже легкий
		\\ \hline
		2
		& Індивідуальні користувачі
		& Потребують простої бібліотеки з інтуітивною структурою та універсальним дизайном
		& Попит серед користувачів Amazon AWS, Microsoft Azure, Digital Ocean
		& Конкуренція із відкритими безкоштовними рішеннями: CloudSim, CloudSim Plus, GridSim
		& Вихід у цей сегмент буде важчим оскільки існуючі рішення більш універсальні та покривають більшу множину задач
		\\ \hline
	\end{tabular}
	\caption{Вибір цільових груп потенційних споживачів}
\end{table}

\begin{table}[H]
	\begin{tabular}
		{|c|p{3cm}|p{3cm}|p{3cm}|p{3cm}|} \hline
		№ п/п
		& Обрана альтернатива розвитку проекту
		& Стратегія розвитку ринку
		& Ключові конкуретноспроможні позиції відповідно до обраної альтернативи
		& Базова стратегія розвитку
		\\ \hline
		
		1
		& Набір основної маси користувачів
		& На початкових етапах давати університетам можливість користуватися безкоштовно продуктом
		& Наявність простого інтерфейсу та швидкої симуляційної системи
		& Просування продукту завдяки науковим публікаціям, які використовують даний програмний продукт
		\\ \hline
		
		2
		& Розвиток симуляційної системи
		& Додавання популярних алгоритмів планування, прикладів аналізу моделей простих математичних задач
		& Розповсюдженість та проста інтеграція для будь-якої популярної платформи
		& Популяризація програмного продукту та введення підписок
		\\ \hline
	\end{tabular}
	\caption{Визначення базової стратегії розвитку}
\end{table}


\begin{table}[H]
	\begin{tabular}
		{|c|p{3cm}|p{3cm}|p{3cm}|p{3cm}|} \hline
		№ п/п
		& Чи є проект першопрохідцем на ринку
		& Чи буде компанія шукати нових споживачів чи забирати існуючих у конкурентів
		& Чи буде компанія копіювати основні характеристики товару конкурента і які?
		& Стратегія конкурентної поведінки
		\\ \hline
		
		1
		& Проект не є першопроходцем
		& Компанія в першу чергу буде забирати споживачів існуючих продуктів
		& Компанія має на меті в спочатку реалізувати аналогічний функціонал як в існуючих продуктах
		& Компанія надає схожий продукт, проте у більш зручній формі та із значно кращою швидкодією
		\\ \hline
	\end{tabular}
	\caption{Визначення базової стратегії конкурентної поведінки}
\end{table}

\begin{table}[H]
	\begin{tabular}
		{|c|p{3cm}|p{3cm}|p{3cm}|p{3cm}|} \hline
		№ п/п
		& Вимоги до товару цільової аудиторії
		& Базова стратегія розвитку
		& Ключові конкурентоспроможні позиції власного стартап-проекту
		& Вибір асоціацій, які мають сформувати комплексну позицію власного проекту (три ключових)
		\\ \hline
		
		1
		& Висока швидкодія, яка дозволить проводити експерименти у системі значно швидше ніж у реальних cloud системах
		& Розробка системи на мові C++ яка дозволить проводити швидкі та точні симуляції
		& Значно вища швидкодія порівняно з аналогами та більш зручний процес аналізу алгоритмів у cloud системах
		& Швидкодія, простота використання, оптимізація паралельних алгоритмів
		\\ \hline
	\end{tabular}
	\caption{Визначення стратегії позиціонування}
\end{table}

\section{Розроблення маркетингової програми стартап-проекту}

\begin{table}[H]
	\centering
	\begin{tabular}{|c|p{4cm}|p{4cm}|p{4cm}|} \hline
		№ п/п
		& Потреба
		& Вигода, яку пропонує товар
		& Ключові переваги перед конкурентами 
		\\ \hline
		
		1
		& Можливість провести експерименти з різними структурами cloud систем перед власне замовленям системи 
		& Товар надає можливість проводити серії симуляцій обчислень, оцінити ефективність структури cloud системи
		& Конкуренти хоч і надають товари з більшим функціоналом, проте симуляція складних задач проходить дуже довго
		\\ \hline
		
		2
		& Збільшення ефективності паралельних алгоритмів
		& Продукт дозволяє симулювати процес обчислень та знаходити оптимальні параметри алгоритма для певної структури cloud системи
		& Конкуренти не надають таких послуг
		\\ \hline
	\end{tabular}
	\caption{Визначення ключових переваг концепції потенційного товару}
\end{table}



\section{Висновки}
Дана магістерська дисертація має перспективи зростання до популярного продукту у сфері хмарних обчислень оскільки наявні аналоги написані на мові Java та мають дуже низьку швидкодію, яка взагалі не дозволяє проводити повномаштабні експерименти з метою оцінки її майбутньої продуктивності. Складнощі в першу чергу можуть бути через недостатню універсальність на початкових етапах релізу продукту та можливу пожвавлену конкуренцію на етапі виходу на ринок. Також конкуренти з великою кількістю ресурсів можуть розпочати розробку власного продукту, що може сильно вплинути на подальший розвиток стартапа. Такі випадки часто трапляються в IT сфері.
\hspace{10pt}

