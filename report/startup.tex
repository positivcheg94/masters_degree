\chapter{Розробка стартап-проекту}
\section{Інформаційна карта проекту}
Назва проекту - ``Efficient task creation for cloud computing". Проект являє собою систему, яка дозволяє правильно розбивати задачі на підзадачі з метою мінімізації часу на їх виконання.
\begin{tabular}
  {|l|p{8cm}|}
\hline

\end{tabular}
\\
Необхідні матеріальні ресурси ресурси:
\begin{enumerate}
\item Керований доступ до панелі керування обчислювальними ресурсами.
\item Юридична особа.
\end{enumerate}
Інтелектуальні ресурси:
\begin{itemize}
  \item математики для аналізу моделей більш складних планувальників
  \item розробники, які зможуть створити алгоритми для правильного розбиття трудоемних задач на підзадачі
  \item спецаліст із маркетингу
  \item операційний розробник із підтримки програмного забезпечення
\end{itemize}
Необхідні фінансові ресурси: гроші, достатні для оплати ринкової зарплати протягом 2-х років, а також для оплати фунціонування одного веб-сайту, маркетинговий бюджет у розмірі 30 000\$.

Проект вирішує проблему ...

Головні цілі проекту:
\begin{itemize}
  \item Аналіз більш складних планувальників з метою побудови їх математичних моделей і знаходження точок рівноваги у грі багатьох користувачів
  \item Виконання трудоемних задач у розподіленому режимі з мінімальним часом
\end{itemize}

Очікувані результати:
\begin{itemize}
  \item Проаналізовані найпопулярніші планувальники чи побудовані алгоритми, що дозволяють правильно оцінювати складності підзадач
  \item Впровадження методу розбиття задач на підзадачі для типових трудоемних проблем
  \item Помітне зменшення часу на виконання трудоемних задач
\end{itemize}

\section{Команда проекту}
\begin{enumerate}
  \item Засновники проекту.
  \item 3 програмістів із знаннями математики.
  \item 2 спеціалісти із хмарних обчислень.
  \item маркетолог
\end{enumerate}

\section{Бізнес-модель CANVAS}

\hspace{10pt}
