\chapter{Керування стартапом проекту}
\section{Інформаційна карта проекту}
Назва проекту - ``Efficient task distribution for cloud computing". Проект являє собою систему, яка дозволяє правильно розбивати математичны задачі на підзадачі з метою мінімізації часу їх виконання у розподіленому середовищі.
\begin{tabular}
  {|l|p{8cm}|}
\hline

\end{tabular}
\\
Необхідні матеріальні ресурси ресурси:
\begin{enumerate}
\item Керований доступ до панелі керування обчислювальними ресурсами.
\item Юридична особа.
\end{enumerate}
Інтелектуальні ресурси:
\begin{itemize}
  \item математики для аналізу моделей більш складних планувальників
  \item розробники, які зможуть створити алгоритми для правильного розбиття трудоемних задач на підзадачі
  \item спецаліст із маркетингу
  \item операційний розробник із підтримки програмного забезпечення
\end{itemize}
Необхідні фінансові ресурси: гроші, достатні для оплати ринкової зарплати протягом 2-х років, а також для оплати фунціонування одного веб-сайту, маркетинговий бюджет у розмірі 30 000\$.

Проект вирішує проблему виконання складних задач, які можна розділити на велику кількість простіших задач, у розподіленому середовищі з мінімізацією часу виконання задач для багатьох користувачів. В першу чергу така проблема може виникнути у дослідницьких центрах, де часто виконуються прості операції проте з неймовірно виликими обсягами даних. Часто такі задачі дуже легко розбивати на більш прості підзадачі, проте не завжди просто оцінити найефективнішу стратегію розбиття.

Головні цілі проекту:
\begin{itemize}
  \item Аналіз більш складних планувальників з метою побудови їх математичних моделей і знаходження точок рівноваги у грі багатьох користувачів
  \item Виконання трудоемних задач у розподіленому режимі з мінімальним часом
\end{itemize}

Очікувані результати:
\begin{itemize}
  \item Проаналізовані найпопулярніші планувальники чи побудовані алгоритми, що дозволяють правильно оцінювати складності підзадач
  \item Впровадження методу розбиття задач на підзадачі для типових трудоемних проблем
  \item Помітне зменшення часу на виконання трудоемних задач
\end{itemize}

\section{Команда проекту}
\begin{enumerate}
  \item Засновники проекту.
  \item 3 програмістів із знаннями математики.
  \item 2 спеціалісти із хмарних обчислень.
  \item маркетолог
\end{enumerate}

\section{Бізнес-модель CANVAS}
Модель CANVAS застосовується для опису поточної та майбутньої стратегії,
для стратегії розвитку новостворених організацій, для переорієнтації
стратегії розвитку діючих організації, розбору існуючої моделі
керування з метою знаходження слабких місць/прогалин в діяльності організації
та пошуку нових точок для зростання. Авторами, творцями бізнес-моделі CANVAS
у 2008 році стали: Олександр Остервальдер –  швейцарський бізнес-теоретик та
Ів Пін’є –  бельгійський вчений і професор інформаційних систем управління.
Після чого модель стрімко поширювалась і зараз застосовується викладачами,
студентами відомих бізнес-шкіл, університетів: Гарвард, Стенфорд, Колумбія, Берклі.\\

Далі розглянемо компоненти бізнес-моделі системи мультимодального текствого пошуку.\\
\begin{enumerate}
	\item Ключові партнери:
	\begin{itemize}
		\item інвестори
		\item рекламні платформи
	\end{itemize}
	\item Ключова діяльність:
	\begin{itemize}
		\item розробка алгоритмів інформаційного пошку
		\item розробка алгоритмів пошуку серед багатовимірних розріджених даних
		\item розробка високоефективних дискових індексів
		\item адапутвання розроброблених алгоритмів для ефективної роботи із диском
		\item розробка веб-платформи для продажу пошукової системи
	\end{itemize}
	\item Дистрибуція:
	\begin{itemize}
		\item підписка
		\item підписка та індивіальна підтримка
		\item довгострокова підписка та розробка розширення функціоналу за вимогами клієнта
	\end{itemize}
	\item Ключові ресурси:
	\begin{itemize}
		\item інтелектуальні ресурси(співробітники)
		\item клієнтська база
	\end{itemize}
	\item Ціннісна пропозиція
	\begin{itemize}
		\item якісний та швидкий пошук при мінімальних витратах ресурсів(трудових та матеріальних)
		\item маштабованість
	\end{itemize}
	\item Відносини з користувачами:
	\begin{itemize}
		\item онлайн-спілкування
		\item онлайн-конференції з великими клієнтами
		\item особисті тренінги
	\end{itemize}
	\item Стуктура витрат:
	\begin{itemize}
		\item підтримка веб-сайту
		\item підтримка делегованих віртальних машин для пошукових систем
		\item маркетинговий бюджет
		\item витрати на розробку
	\end{itemize}
	\item Стуктура доходів
	\begin{itemize}
		\item підписка 
		\item корпоративні тренінги
	\end{itemize}
\end{enumerate}
\section{Аналіз ринкових можливостей запуску стартап-проекту}
\begin{table}
	\begin{center}
		\begin{tabular}
			{|l|p{8cm}|p{5cm}|}\hline
			\bf{№} & \bf{Показники стану ринку(найменування)} & \bf{Характеристика} \\ \hline
			1 & Кількість головних гравців, од. & 5 \\ \hline
			2 & Загальний обсяг гравців, \$ & 500 млрд. \\ \hline
			3 & Динаміка ринку & Попит зростає, проте пропозиція не збільшується\\ \hline
			4 & Обмеження для входу на ринок& Наявність якісної технології\\ \hline
			5 & Специфічні вимоги до стандартизації та сертифікації& Відсутні \\ \hline
			6 & Середня норма рентабельності в галузі & 100\% \\ \hline
		\end{tabular}
	\end{center}
	\caption{Попередня характеристика потенційного ринку стартап-проекту}
\end{table}


\begin{table}
	\begin{tabular}
		{|l|p{3.5cm}|p{3.5cm}|p{3.5cm}|p{3.5cm}|} \hline
		\bf{№} & \bf{Потреба, що формує ринок} & \bf{Цільова аудиторія(цільові сегменти ринку)} & 
		\bf{Відмінності у поведінці різних потенційних цільових груп} & \bf{Вимоги користувачів до товару} \\ \hline
		
		1 & Необхідність пошуку релевантних товарів & Інтернет-магазини & Різні магазини мають різні кількості різних типів товарів & Необхідно знаходити різноманітні категорії товарів
		(забезпечувати варіативність пошукових результатів)\\ \hline
		
		2 & Необхідність пошуку релевантних записів& Публічні та приватні блоги, портали новин, інформаційні сайти & Різні ресурси мають різну структуру даних & Пошук повинен враховувати інтереси користувачів а також застарілість чи новизну даних\\ \hline
	\end{tabular}
	\caption{Характеристика потенційних клієнтів}
\end{table}

\begin{table}
	\begin{tabular}
		{|l|p{4cm}|p{4cm}|p{4cm}|} \hline
		№ & Фактор & Зміст загрози & Можлива реакція компанії \\ \hline
		1 & Здорожчання послуг хармарних обчислень& Здорожчання цін на DigitalOcean там Amazon приведе до здорожчання тарифів і можливих втрат, пов'язаних із довготривалими підписками & Можлива реакція компанії \\ \hline
		№ & Втрата ключових клєнтів& Приводить до зменшення доходів & Дізнатися причину втрати \\ \hline
	\end{tabular}
	\caption{Фактори загроз}
\end{table}

\begin{table}
	\begin{tabular}
		{|l|p{4cm}|p{4cm}|p{4cm}|} \hline
		№ & Фактор & Зміст можливості & Можлива реакція компанії \\ \hline
		1 & Спіробітницство із розвинутими Інтернет-Магазинами & Можливість отримати дохідного клієнта & Підлаштовування платформи під вимоги клієнта \\ \hline
		2 & Публікація нового стандарту в області пошуку & Можливість швидко адаптувати останнє дослідження & Покращення якості пошку \\ \hline
	\end{tabular}
	\caption{Фактори можливостей}
\end{table}


\begin{table}
	\begin{tabular}
		{|l|p{4cm}|p{4cm}|p{4cm}|} \hline
		№ & Особливості конкуретного середовища & В чоему проявляється дана характеристика & Вплив на діяльність підприємства (можливі дії компанії, що бути конкуретноспроможною) \\ \hline
		1 & Олігополія & Існує 2 відкриті пошукові системи: Apache Solr, Elastic Search & Характеризують мінамльно допустиму якість продукту \\ \hline
		2 & Міжнародний рівень конкуретної боротьби & Цифрові продукту інформаційного пошуку не мають кородонів & Серйозне ставлення до стійкості системи, рівня безпеки та якості пошуку \\ \hline
		3 & Галузева конкуренція & Конкуренція проходить у галузі інформаційного пошуку & Бути золотим стандартом у галузі \\ \hline
		4 & Товарно-видова конкуренція за видами товарів & Наявність функціоналу визначає потенціний розмір бази користувачів & Постійне спостереження за потребами клієнтів \\ \hline
		5 & За характером конкурентних перваг: цінова/нецінова & Якість пошуку є конкуретною перевагою, водночас товар може обходитись дешевше, ніж альтернативні варіанти & Розширення функціоналу, вдосконалення математичних моделей при сталих цінах \\ \hline
		6 & За інтенсивністю - марочна конкуренція & Ці не можуть бути тривалий час меншими за вартість хмарних обчислень & Вихід на нових великих користувачів \\ \hline
	\end{tabular}
	\caption{Ступеневий аналіз конкуренції на ринку}
\end{table}

\begin{table}
	\begin{tabular}
		{|l|p{10cm}|} \hline
		\bf{Назва характеристики} & \bf{Характеристика} \\ \hline
		Прямі конкуренти & vespa.ai \\ \hline
		Потенційні конкуренти & Goolgle, Yandex, Microsoft, ElasticSearch, Amazon, Solr \\ \hline
		Постачальники & Amazon, Digital Ocean, Google \\ \hline
		Клієнти & Home Depot, StackOverflow, The New York Times, The Economist \\ \hline
		Товари-замінники & Solr, Elastic Search, MySQL, PostgreSQL \\ \hline
	\end{tabular}
	\caption{Аналіз конкуренції за М. Портером (Складові аналізу)}
	\begin{tabular}
		{|l|p{10cm}|} \hline
		\bf{Назва характеристики} & \bf{Характеристика} \\ \hline
		Прямі конкуренти & Прямі конкуренти - дрібні компанії \\ \hline
		Потенційні конкуренти & Потенційні конкуренти - компанії-гіганти \\ \hline
		Постачальники & Постачальники хмарних потужностей - потенційні конкуренти\\ \hline
		Клієнти & Для клієнтів якісний пошук - ядро їхньої системи \\ \hline
		Товари-замінники & Товари-замінники працюють повільно та демонструють низьку якість пошуку \\ \hline
	\end{tabular}
	\caption{Аналіз конкуренції за М. Портером (Висновки)}
	
\end{table}

\begin{table}
	\begin{tabular}
		{|l|p{7cm}|p{7cm}|} \hline
		№ & Фактор конкурентноспроможності & Обґрунтування (чинники, що роблять фактор для порівння конкурентних проектів значущим) \\ \hline
		1 & Застосування алгоритмів машинного начання для пошуку & Дає можливість розуміти семантику пошукового запиту \\ \hline
		2 & Лінійна маштобованість & Дає можливість адаптуватися до зростаючих об'ємів даних \\ \hline
		3 & Низька затримка у 99\%-х випадків  & Асинхронна модель дозволяє рівномірно розподіляти затримки на по всім з'єднанням \\ \hline
		4 & Динамічна структура функції релевантності & Дає можливість адаптувати пошукову систему під універсальні потреби \\ \hline
	\end{tabular}
	\caption{Обґрунтування факторів конкурентноспроможності}
\end{table}
% Please add the following required packages to your document preamble:
% \usepackage{multirow}
\begin{table}[]
	\centering
	\begin{tabular}{|c|p{8cm}|p{1cm}|c|c|c|c|c|c|c|}
		\hline
		\multirow{2}{*}{\#} & \multirow{2}{*}{Фактор конкурентноспроможності}                 & \multirowcell{2}{Бали\\ 1-20} & \multicolumn{7}{c|}{Рейтинг відносно vespa.ai} \\ \cline{4-10}
		
		&                                                                 &                            &  -3 & -2  & -1  & 0 & 1 & 2 & 3 \\ \hline
		1                   & Застосування алгоритмів машинного начання для пошуку            & 20                         &     &     &     &   &   &   & + \\ \hline
		2                   & Лінійна маштобованість                                          & 20                         &     &     &     &   &   &   & + \\ \hline
		3                   & Низька затримка у 99\%-х випадків                               & 20                         &     &     &     &   &   &   & + \\ \hline
		4                   & Динамічна структура функції релевантності                       & 20                         &     &     &     &   &   &   & + \\ \hline
	\end{tabular}
	\caption{Порівняльний аналіз сильних та слабких сторін}
\end{table}
\pagebreak
SWOT-аналіз старартап-проекту:
\begin{enumerate}
	\item Cильні сторони:
	\begin{enumerate}
		\item швидкодія
		\item простота налаштування розподіленого середовища
		\item врахування семантики документів
		\item мыльтимодальны метрики релевантносты
		\item ефективне стиснення інформації
		\item малі затримки в обробці запитів
		\item варіативності релевантних пошукових результатів
	\end{enumerate}
	\item Слабкі сторони:
	\begin{enumerate}
		\item відсутність можливості додавання програмованих розширень
		\item підтримка однієї платформи
		\item необхідність перевантажувати систему для застосування деяких нових налаштувань
	\end{enumerate}
	\item Можливості:
	\begin{enumerate}
		\item можливість лінійно маштабуватися відносно кількості користувачів
		\item незалежність швидкодії системи від кількості клієнтів
		\item можливість вибірково встановлювати оновлення пошукової системи
		\item можливість покращувати систему за рахунок перехресного використання даних
		\item можливість стати стандартом де-факто в області мультимодального пошуку
		\item можливість розширити асортимент наданих послуг
	\end{enumerate}
	\item Загрози:
	\begin{enumerate}
		\item здорожчання послуг постачальника віртуальних машин
		\item поява конкурентів
		\item втрата одного або кількох найбільших користувачів
	\end{enumerate}
\end{enumerate}

\begin{table}
	\begin{tabular}{|c|p{4cm}|p{4cm}|p{4cm}|} \hline
		№ & Альтернатива (орієнтований комплекс заходів) ринкової поведінки & Ймовірність отримання ресусрів & Строки реалізації \\ \hline
		1 & Розробка модуля для популярних СУБД, веб-сайту та пошукової системи. & 0.5 & 1 рік \\ \hline
		2 & Розробка модуля для популярних СУБД, веб-сайту, клієнтського API для популярних мов програмування та пошукової системи. & 0.7 & 1.5 роки \\ \hline
		3 & Розробка розширення для Postgres та MySQL у вигляді динамічної бібліотеки, що об'єднує базу та пошукову системи. & 0.7 & 1.5 роки \\ \hline
	\end{tabular}
	\caption{Альтернативи ринкового впровадження стартап-проекту}
\end{table}
\section{Розробка ринкової стратегії проекту}
\begin{table}
	\begin{tabular}
		{|c|p{3cm}|p{3cm}|p{3cm}|p{3cm}|p{3cm}|}
		№ & Опис профілю цільової групи потенційних клієнтів & Готовність споживачів сприйти продукт & Орієнтовний помит цільової групи (сегменти)  & Інтенсивність конкуренції в сегменті & Простота входу у сегмент \\ \hline
		1 & Веб-сайти середнього розміру, що потребують якісного пошуку & Потребують простих засобів для інтеграції & Попит серед веб-сайтів наступних типів: Інтернет-Магазини, інформаційні портали, каталоги продукції & Конкуренція із відкритими безкоштовними рішеннями \\
		2 & Крупні веб-сайти, що потребують якісного пошуку & Потребують простих ефективних інтерфейсів для синхронізації даних, потребують підтримки шардування & Попит серед веб-сайтів наступних типів: Інтернет-Магазини, інформаційні портали, каталоги продукції & Конкуренція із відкритими безкоштовними рішеннями \\
	\end{tabular}
	\caption{Вибір цільових груп потенційних споживачів}
\end{table}

\begin{table}
	\begin{tabular}
		{|c|p{3cm}|p{3cm}|p{3cm}|p{3cm}|} \hline
		№ & Обрана альтернатива розвитку проекту & Стратегія розвитку ринку & Ключові конкуретноспроможні позиції відповідно до обраної альтернативи & Базова стратегія розвитку \\ \hline
		1 & Набір критичної маси функціоналу, необхідної для виходу на ринок & На початкових етапах охоплювати ринок за рахунок користувачів CMS, надалі - розширюватись на основі ідеального профілю замовників & Наявність власної системи інтелектуального текстового пошуку & Різні тарифні плани націлені на різні групи користувачів, найбільш дешевий варіант повинен продаватися за собівартістю, оскільки його мета - змусити аудиторію користуватися сервіом. \\ \hline
		2 & Розвиток API та підтримка нових видів інтеграції сервісу & Розробляються розширення для популярних СУБД для серврної частини та JavaScript API - для клієнтської. При цьому спілкування із пошукової систему здійснюється за допомогою авторизації через OAUTH2 & Розповсюдженість та проста інтеграція для будь-якої популярної платформи & Додаткова плата за шардування даних. \\ \hline
	\end{tabular}
	\caption{Вибір цільових груп потенційних споживачів}
\end{table}


\begin{table}
	\begin{tabular}
		{|c|p{3cm}|p{3cm}|p{3cm}|p{3cm}|} \hline
		№ & Чи є проект першопрохідцем на ринку & Чи буде компанія шукати нових споживачів чи забирати існуючих у конкурентів &  Чи буде компанія копіювати основні характеристики товару конкурента і які? & Стратегія конкурентної поведінки \\ \hline
		1 & Проект є одним із першопрохідців на ринку & Компанія будет як шукати нових споживачів, так і переносити існуючі рішення з ElasticSearch на реалізовану платформу & Компанія будет оглядатися на існуючі відкриті рішення, такі як Elastic Search та Apache Solк & Компанія буде пропонуватиме готовий сервіс, який конкурує за рахунок якості пошуку та швидкодії \\ \hline
	\end{tabular}
	\caption{Визначення базової стратегії конкурентної поведінки}
\end{table}
\section{Висновки}
Дана магістерська дисертація має перспективи зростання до повноцынної компанії за рахунок поступового охоплення ринку інформаційного пошуку, витісняючи відкриті рішення, побудовані на базі Apache Lucene, оскыльки витрати на обслуговування серверыв ыз таким програмним забезпеченням є більшими за витрати та додану вартість розробленої системи.
\hspace{10pt}

