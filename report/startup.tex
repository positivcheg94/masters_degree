\chapter{Стартапи}
\section{Інформаційна карта проекту}
Назва проекту - ``Multimodal Text Search System". Проект являє собою пошуку систему, яка може представлятися у якості сервісу для кінцевих користувачів.
\begin{tabular}
  {|l|p{8cm}|}
\hline

\end{tabular}
\\
Необхідні матеріальні ресурси ресурси:
\begin{enumerate}
\item Керований доступ до панелі керування обчислювальними ресурсами.
\item Юридична особа.
\end{enumerate}
Інтелектуальні ресурси:
\begin{itemize}
  \item програмісти ядра системи із знанням математики
  \item веб-розробники
  \item спецаліст із маркетингу
  \item співробітники служби пітримки
  \item відділ лінгвістики
  \item дизайнер
  \item операційний розробник із підтримки програмного забезпечення
\end{itemize}
Необхідні фінансові ресурси: гроші, достатні для оплати ринкової зарплати протягом 3-х років, а також для оплати фунціонування одного веб-сайту, маркетинговий бюджет у розмірі 50 000\$.

Проект вирішує проблему, яка характеризується тим, що сучасні бази даних не надають достатніх можливостей для
повноцінного текстового пошуку. Існуючі спеціалізовані системи не надають можливості працювати із мультимодальними даними, до того ж, оскільки вони написані із використанням мов програмування,
які не мають повноцінного доступу до системних викликів операціної системи та високошвидкісних процесорних операцій та мають вбудоване автоматичне збирання мусору, таким чином, реалізації
мають в десятки разів меншу швидкодію, ніж можливо. Таким чином, оплата стороннього сервісу може виявитися дешевшою за плату, необхідну для функціонування самостійно розгорнутої відкрітої
пошукової системи.

Головні цілі проекту:
\begin{itemize}
  \item Побудова системи мультимодального пошуку на основі ефективних алгоритмів текстового пошуку.
  \item Максимальна утилізація наявних обчислювальних ресурсів.
  \item Якісні метрики отриманої системи мають переважити існуючі системи.
  \item Робастний пошук текстових документів.
\end{itemize}

Очікувані результати:
\begin{itemize}
  \item Досягнуто усі головні цілі проекту.
  \item Наявність передплачених річних та більш довгострокових передплат від користувачів ресурсу.
  \item Постійний притік нових користувачів.
  \item Розроблені технології дозволяють вдосконалювати пошукові метрики без деградації швидкодії.
  \item Система оновлень не наносить шкоди функціонуванню запущених систем.
\end{itemize}

\section{Команда проекту}
\begin{enumerate}
  \item Засновники проекту.
  \item 5 програмістів із знання математики.
  \item 3 спеціалісти із пошукових метрик.
  \item 2 веб-розробника.
  \item маркетолог
\end{enumerate}
Завданням засновнику проекту є контроль за якістю програмного продукту, підтримуваністю кодової бази, правильним використанням обчислювальних ресурсів хмари, у які зберігаються віртуальні машини.

Завданнями програмістів є постійне вдосконалення швидкодії та підртримуваності існуючої системи, а також вдосконалення реалізованих алгоритмів, а також за необхідності - заміна алгоритмів на
більш якісні.

Завданнями спеціалістів із пошукових метрик є вдосконалення функції провдоподібності, яка застосувується при моделюванні тем, інформаційному пошку.

Завданням веб-зробників є розробка користувацьких інтерфейсів для роботи із пошуковою системою, безпечна обробка Інтернет-платежів.

Завданням маркетолога є пошук цільової аудиторії проекту, формулювання портрету ідеального користувача, аналіз вартості та ефективності рекламних кампаній,озробка планів по освоєнню маркетингових бюджетів,
огляд створених тарифних планів та доповленостей використання ресурсів.

\section{Бізнес-модель CANVAS}
Модель CANVAS застосовується для опису поточної та майбутньої стратегії,
для стратегії розвитку новостворених організацій, для переорієнтації
стратегії розвитку діючих організації, розбору існуючої моделі
керування з метою знаходження слабких місць/прогалин в діяльності організації
та пошуку нових точок для зростання. Авторами, творцями бізнес-моделі CANVAS
у 2008 році стали: Олександр Остервальдер –  швейцарський бізнес-теоретик та
Ів Пін’є –  бельгійський вчений і професор інформаційних систем управління.
Після чого модель стрімко поширювалась і зараз застосовується викладачами,
студентами відомих бізнес-шкіл, університетів: Гарвард, Стенфорд, Колумбія, Берклі.\\

Далі розглянемо компоненти бізнес-моделі системи мультимодального текствого пошуку.\\
\begin{enumerate}
  \item Ключові партнери:
  \begin{itemize}
          \item інвестори
          \item рекламні платформи
        \end{itemize}
      \item Ключова діяльність:
\begin{itemize}
  \item розробка алгоритмів інформаційного пошку
  \item розробка алгоритмів пошуку серед багатовимірних розріджених даних
  \item розробка високоефективних дискових індексів
  \item адапутвання розроброблених алгоритмів для ефективної роботи із диском
  \item розробка веб-платформи для продажу пошукової системи
\end{itemize}
\item Дистрибуція:
\begin{itemize}
  \item підписка
  \item підписка та індивіальна підтримка
  \item довгострокова підписка та розробка розширення функціоналу за вимогами клієнта
\end{itemize}
\item Ключові ресурси:
  \begin{itemize}
    \item інтелектуальні ресурси(співробітники)
    \item клієнтська база
  \end{itemize}
\item Ціннісна пропозиція
  \begin{itemize}
    \item якісний та швидкий пошук при мінімальних витратах ресурсів(трудових та матеріальних)
    \item маштабованість
  \end{itemize}
\item Відносини з користувачами:
  \begin{itemize}
    \item онлайн-спілкування
    \item онлайн-конференції з великими клієнтами
    \item особисті тренінги
  \end{itemize}
\item Стуктура витрат:
  \begin{itemize}
    \item підтримка веб-сайту
    \item підтримка делегованих віртальних машин для пошукових систем
    \item маркетинговий бюджет
    \item витрати на розробку
  \end{itemize}
\item Стуктура доходів
  \begin{itemize}
    \item підписка 
    \item корпоративні тренінги
  \end{itemize}
\end{enumerate}
\section{Аналіз ринкових можливостей запуску стартап-проекту}
\begin{table}
  \begin{center}
    \begin{tabular}
        {|l|p{8cm}|p{5cm}|}\hline
        \bf{№} & \bf{Показники стану ринку(найменування)} & \bf{Характеристика} \\ \hline
        1 & Кількість головних гравців, од. & 5 \\ \hline
        2 & Загальний обсяг гравців, \$ & 500 млрд. \\ \hline
        3 & Динаміка ринку & Попит зростає, проте пропозиція не збільшується\\ \hline
        4 & Обмеження для входу на ринок& Наявність якісної технології\\ \hline
        5 & Специфічні вимоги до стандартизації та сертифікації& Відсутні \\ \hline
        6 & Середня норма рентабельності в галузі & 100\% \\ \hline
    \end{tabular}
  \end{center}
  \caption{Попередня характеристика потенційного ринку стартап-проекту}
\end{table}


\begin{table}
  \begin{tabular}
  {|l|p{3.5cm}|p{3.5cm}|p{3.5cm}|p{3.5cm}|} \hline
    \bf{№} & \bf{Потреба, що формує ринок} & \bf{Цільова аудиторія(цільові сегменти ринку)} & 
    \bf{Відмінності у поведінці різних потенційних цільових груп} & \bf{Вимоги користувачів до товару} \\ \hline

    1 & Необхідність пошуку релевантних товарів & Інтернет-магазини & Різні магазини мають різні кількості різних типів товарів & Необхідно знаходити різноманітні категорії товарів
    (забезпечувати варіативність пошукових результатів)\\ \hline

    2 & Необхідність пошуку релевантних записів& Публічні та приватні блоги, портали новин, інформаційні сайти & Різні ресурси мають різну структуру даних & Пошук повинен враховувати інтереси користувачів а також застарілість чи новизну даних\\ \hline
  \end{tabular}
  \caption{Характеристика потенційних клієнтів}
\end{table}

\begin{table}
  \begin{tabular}
    {|l|p{4cm}|p{4cm}|p{4cm}|} \hline
    № & Фактор & Зміст загрози & Можлива реакція компанії \\ \hline
    1 & Здорожчання послуг хармарних обчислень& Здорожчання цін на DigitalOcean там Amazon приведе до здорожчання тарифів і можливих втрат, пов'язаних із довготривалими підписками & Можлива реакція компанії \\ \hline
    № & Втрата ключових клєнтів& Приводить до зменшення доходів & Дізнатися причину втрати \\ \hline
  \end{tabular}
  \caption{Фактори загроз}
\end{table}

\begin{table}
  \begin{tabular}
    {|l|p{4cm}|p{4cm}|p{4cm}|} \hline
    № & Фактор & Зміст можливості & Можлива реакція компанії \\ \hline
    1 & Спіробітницство із розвинутими Інтернет-Магазинами & Можливість отримати дохідного клієнта & Підлаштовування платформи під вимоги клієнта \\ \hline
    2 & Публікація нового стандарту в області пошуку & Можливість швидко адаптувати останнє дослідження & Покращення якості пошку \\ \hline
  \end{tabular}
  \caption{Фактори можливостей}
\end{table}


\begin{table}
  \begin{tabular}
    {|l|p{4cm}|p{4cm}|p{4cm}|} \hline
    № & Особливості конкуретного середовища & В чоему проявляється дана характеристика & Вплив на діяльність підприємства (можливі дії компанії, що бути конкуретноспроможною) \\ \hline
    1 & Олігополія & Існує 2 відкриті пошукові системи: Apache Solr, Elastic Search & Характеризують мінамльно допустиму якість продукту \\ \hline
    2 & Міжнародний рівень конкуретної боротьби & Цифрові продукту інформаційного пошуку не мають кородонів & Серйозне ставлення до стійкості системи, рівня безпеки та якості пошуку \\ \hline
    3 & Галузева конкуренція & Конкуренція проходить у галузі інформаційного пошуку & Бути золотим стандартом у галузі \\ \hline
    4 & Товарно-видова конкуренція за видами товарів & Наявність функціоналу визначає потенціний розмір бази користувачів & Постійне спостереження за потребами клієнтів \\ \hline
    5 & За характером конкурентних перваг: цінова/нецінова & Якість пошуку є конкуретною перевагою, водночас товар може обходитись дешевше, ніж альтернативні варіанти & Розширення функціоналу, вдосконалення математичних моделей при сталих цінах \\ \hline
    6 & За інтенсивністю - марочна конкуренція & Ці не можуть бути тривалий час меншими за вартість хмарних обчислень & Вихід на нових великих користувачів \\ \hline
  \end{tabular}
  \caption{Ступеневий аналіз конкуренції на ринку}
\end{table}

\begin{table}
  \begin{tabular}
    {|l|p{10cm}|} \hline
    \bf{Назва характеристики} & \bf{Характеристика} \\ \hline
    Прямі конкуренти & vespa.ai \\ \hline
    Потенційні конкуренти & Goolgle, Yandex, Microsoft, ElasticSearch, Amazon, Solr \\ \hline
    Постачальники & Amazon, Digital Ocean, Google \\ \hline
    Клієнти & Home Depot, StackOverflow, The New York Times, The Economist \\ \hline
    Товари-замінники & Solr, Elastic Search, MySQL, PostgreSQL \\ \hline
  \end{tabular}
  \caption{Аналіз конкуренції за М. Портером (Складові аналізу)}
  \begin{tabular}
    {|l|p{10cm}|} \hline
    \bf{Назва характеристики} & \bf{Характеристика} \\ \hline
    Прямі конкуренти & Прямі конкуренти - дрібні компанії \\ \hline
    Потенційні конкуренти & Потенційні конкуренти - компанії-гіганти \\ \hline
    Постачальники & Постачальники хмарних потужностей - потенційні конкуренти\\ \hline
    Клієнти & Для клієнтів якісний пошук - ядро їхньої системи \\ \hline
    Товари-замінники & Товари-замінники працюють повільно та демонструють низьку якість пошуку \\ \hline
  \end{tabular}
  \caption{Аналіз конкуренції за М. Портером (Висновки)}

\end{table}

\begin{table}
  \begin{tabular}
    {|l|p{7cm}|p{7cm}|} \hline
    № & Фактор конкурентноспроможності & Обґрунтування (чинники, що роблять фактор для порівння конкурентних проектів значущим) \\ \hline
    1 & Застосування алгоритмів машинного начання для пошуку & Дає можливість розуміти семантику пошукового запиту \\ \hline
    2 & Лінійна маштобованість & Дає можливість адаптуватися до зростаючих об'ємів даних \\ \hline
    3 & Низька затримка у 99\%-х випадків  & Асинхронна модель дозволяє рівномірно розподіляти затримки на по всім з'єднанням \\ \hline
    4 & Динамічна структура функції релевантності & Дає можливість адаптувати пошукову систему під універсальні потреби \\ \hline
  \end{tabular}
  \caption{Обґрунтування факторів конкурентноспроможності}
\end{table}
% Please add the following required packages to your document preamble:
% \usepackage{multirow}
\begin{table}[]
  \centering
  \begin{tabular}{|c|p{8cm}|p{1cm}|c|c|c|c|c|c|c|}
    \hline
    \multirow{2}{*}{\#} & \multirow{2}{*}{Фактор конкурентноспроможності}                 & \multirowcell{2}{Бали\\ 1-20} & \multicolumn{7}{c|}{Рейтинг відносно vespa.ai} \\ \cline{4-10}

                        &                                                                 &                            &  -3 & -2  & -1  & 0 & 1 & 2 & 3 \\ \hline
    1                   & Застосування алгоритмів машинного начання для пошуку            & 20                         &     &     &     &   &   &   & + \\ \hline
    2                   & Лінійна маштобованість                                          & 20                         &     &     &     &   &   &   & + \\ \hline
    3                   & Низька затримка у 99\%-х випадків                               & 20                         &     &     &     &   &   &   & + \\ \hline
    4                   & Динамічна структура функції релевантності                       & 20                         &     &     &     &   &   &   & + \\ \hline
  \end{tabular}
  \caption{Порівняльний аналіз сильних та слабких сторін}
\end{table}

SWOT-аналіз старартап-проекту
Cильны сторони:
\begin{enumerate}
  \item швидкодія
  \item простота налаштування розподіленого середовища
  \item розуміння семантики документів
\end{enumerate}

Слабкі сторони:
\begin{enumerate}
  \item відсутність можливості програмувати розширення
  \item підтримка однієї платформи
  \item необхідність перевантажувати систему для застосування деяких нових налаштувань
\end{enumerate}
\hspace{10pt}
\cite{Baird1992}
