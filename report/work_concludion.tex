\likechapter{Висновки по роботі та перспективи подальших досліджень}

У роботі проведено аналіз задачі множення матриць методом побудови потокової моделі для оцінки часу виконання задач одного користувача, а далі узагальнено для двох користувачів. Також чітко показано, що складність обчислень не збільшується в залежності від кількості ОВ, їх потужностей та інших параметрів ХС. Показано як впливає розмір розбиття на величину штрафів за пересилки. Також ця методологія може бути використана для аналізу інших задач, наприклад основних сценаріїв стантарта BLAS (Basic Linear Algebra Subprograms), які є основою усіх наукових обчислень навіть у великих проектах.

Розроблено СПЗ, яке дозволяє дуже швидко емулювати процес множення матриць з використанням одного із чотирьх статичних планувальників: minmin, minmax, maxmin та maxmax. Також це СПЗ універсальне та допускає розширення, удосконалення з можливістю переходу у реальний продукт, на базі якого можливо буде емпірично визначати параметри розбиття чи перевірки інших математичних моделей задач.

Результати даної роботи представлені у журналі "Проблеми програмування" - Ігнатенко О.П., Одобеску В.Я.: Теоретико-ігровий агаліз планувальників у багатопроцесорних системах. Імітаційна модель / Проблеми програмування. — 2018. — № 2-3. (прийнята до друку).

У подальшому також можна розглянути двох гравців як незалежні штучні інтелекти, які навчаються вибирати розрізання, та мають єдиний сигнал зворотнього зв'язку - час отримання усіх своїх задач. Таким чином можна дослідити деякі алгоритми навчання з підкріпленням, оскільки це добре підходить під концепцію гри де виграші на момент $t$ якраз і будуть зворотнім сигналом. Основними алгоритмами у цій сфері є: Q-learning, SARSA, DQN, DDPG.