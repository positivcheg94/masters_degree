\begin{frame}{Проведення експериментів}
	\manimate
	
	Для проведення експериментів існують спеціальні пакети на мові Java - CloudSim, GridSim, DARTCSIM та інші.
	
	Проблема їх усіх в тому, що вони базуються на мові Java та працюють дуже повільно у випадку великої кількості симуляцій.
	
	Тому була розроблена власне симуляційна програма.
\end{frame}

\begin{frame}{Графіки часів для різних розмірів матриць (один користувач)}
	\manimate
	
	\includegraphics[width=0.8\linewidth]{im/one_user_different_N}
\end{frame}

\begin{frame}{Графіки часів для різних розмірів матриць}
	\manimate
	
	\includegraphics[width=0.9\linewidth]{im/one_user_different_proc}
\end{frame}

\begin{frame}{Графіки часів для різної кількості обчислювальних вузлів (один користувач)}
	\manimate

	\includegraphics[width=0.7\linewidth]{im/two_users_fixed_first}
\end{frame}

\begin{frame}{Графіки часів для двох користувачів}
	\manimate

	\includegraphics[width=0.6\linewidth]{im/two_users_surface_plot_20_400}
\end{frame}

\begin{frame}{Графіки часів для двох користувачів}
	\manimate
	
	\centering
	\includegraphics[width=0.4\linewidth]{im/two_users_surface_plot_150_350_bot}
	
	\includegraphics[width=0.4\linewidth]{im/two_users_surface_plot_150_350_top}
\end{frame}

\begin{frame}{Графіки часів для двох користувачів}
	\manimate
	
	\centering
	\includegraphics[width=0.6\linewidth]{im/nash_strategy_together}
\end{frame}

\begin{frame}{Застосування методу оптимізації для спуску до оптимальної точки}
	\manimate
	
	\centering
	\includegraphics[width=0.6\linewidth]{im/box}

\end{frame}

\begin{frame}{Застосування методу оптимізації для спуску до оптимальної точки}
	\manimate
	
	\centering
	
	\begin{table}[H]
		\tiny
		\begin{tabular}{|l|l|l|l|l|}
			\hline
			№ & Початкова стратегія & Початковий час 	& Кінцева стратегія & Кінцевий час
			\\ \hline
			1.& (2,2)				& 6670.72055426     &  (1250, 2000) 	& 11.0248749
			\\ \hline
			2.& (10000,10000)		& 14.65571875   	&  (5000,10000)		& 13.7184062
			\\ \hline
			3.& (5,25)				& 1335.36618371		&  (1000,1250)		& 11.0248749
			\\ \hline
			4.& (25,5)				& 1604.09384557		&  (1000,1250)		& 11.0248749
			\\ \hline
			5.& (25,5)				& 1604.09384557		&  (1000,1250)		& 11.0248749
			\\ \hline
			6.& (625,40)			& 184.069188072		&  (1000,1250)		& 11.0248749
			\\ \hline
			
		\end{tabular}
	\end{table}

\end{frame}






