\begin{frame}{Висновки}
	\manimate
	\begin{itemize}
		\item У роботі побудована модель задачі множення матриць блочно у розподіленому середовищі. Проаналізовані штрафи за дрібність розбиття.
		
		\item Проведено пошук рівноваг Неша.
		
		\item Розглянуло альтернативний підхід до пошуку оптимальної точки.
		
	\end{itemize}

\end{frame}

\begin{frame}{Шляхи подальшого розвитку}
	\manimate
	
	У подальшому можна розглянути інші стратегії розбиття задачі множення матриць на підзадачі та більш складні структури Cloud середовищ.
	
	Також слід розглянути інші програми із стандарту BLAS, оскільки вони є основою усіх наукових проектів.
\end{frame}

\setbeamertemplate{logo}{\includegraphics[height=0.2\textheight]{ipsa3.png}}
\begin{frame}
	\manimate
	\centering
	\Large Дякую за увагу.
\end{frame}