\likechapter{Вступ}

Хмарні обчислення на даний момент вважається чимось достатньо простим да доступним, деякі компанії навіть дозволяють безкоштовно використовувати певну кількість ресурсів, також університетам надаються ресурси для складних обчислень, які просто не можливо провести за допомоги звичайних комп'ютерів.

Планувальник у хмарних обчисленнях це те, що управляє самим процесом розподілення задач на обчислювальних вузлах, які виділені для розподіленої системи чи хмари. Саме від стратегії розподілення задач залежить як швидко користувачі отримають результати, наскільки справедливо буде розподілений час обчислень між багатьма користувачами та інші параметри.

Чимало досліджень виконано з метою проектування найефективнішого планувальника, проте такого ще не існує. Частіше за все для кожного з планувальників можна знайти перелік параметрів які він оптимізує. І все ще немає такого планувальника, який буде кращим за інший по всім параметрам та бути справедливим у середовищі з багатьма користувачами.

Основною метою цієї роботи є не розробка найкращого планувальника, а аналіз звичайних планувальників та характеристик задач, які будуть краще всього розкривати потенціал вибраного планувальника. Також у роботі проведений ігровий аналіз процесу обчислення добутку двох матриць у розподіленому середовищі як гри двох користувачів із різними стратегіями розрізання матриці на блоки для паралельного обчислення добутку. Особливістю такої проблеми є те, що сам процес множення матриць при їх великих розмірах може займати дуже багато часу і тому в першчу чергу потрібно розробити симуляційне програмне забезпечення для більш швидкого аналізу моделей та алгоритмів.