\section{Імплементація імітаційної моделі}

Імітаційна модель дозволяє симулювати процес множення матриць у розподіленому середовищі враховуючи як час на власне обчислення так і передачу даних.

Розглянемо задачу множення двох $N*N$ матриць M1 та M2. Позначимо за $n$ кількість рядків матриці М1 та відповідно стовбців матриці М2. Таким чином отримуємо 2 підматриці $n*N$ та $N*n$ які і формують одну задачу, що буде відіслана планувальнику. Таких задач буде $ \floor{ \frac{N}{n} } \times \floor{ \frac{N}{n} } $. Проте в обох матрицях М1 та М2 у випадку якщо n не дільник N буде залишок рядків М1 та стовбців М2 відповідно. Позначимо розмір залишка за  . Тому до основних задач ще потрібно додати задачі множення матриць   та  , матриць   та   і задачу множення  та  .