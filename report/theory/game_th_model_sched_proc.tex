\section{Теоретико-ігрове моделювання процесів планування}

В роботах [4, 6] була побудована аналітична модель алгоритму множення матриць, яка дозволяє оцінити оптимальний за часом розв’язок. 

Коротко нагадаємо основні визначення. Нехай система складається з $m$ обчислювальних елементів та кожен з них характеризується швидкістю роботи $p_i , i=1,\ldots,m$ – тобто кількістю операцій з плаваючою точкою за секунду, які він може здійснити. Процесори з’єднані лініями зв’язку з планувальником, який передає задачі та приймає від них результат. Будемо вважати, що лінії зв’язку ідентичні та мають швидкість передачі даних $q$ та затримку l. 
Будемо вважати, що виконуються наступні припущення
\begin{itemize}
	\item Всі процесори починають роботу одночасно
	\item Планувальник здійснює призначення миттєво
\end{itemize}

Нехай задані дві квадратні матриці розмірності $N \times N$ , результат множення яких необхідно обчислити. При використанні блочного алгоритму користувач задає розмір блоку $n$ , в результаті чого формуються $k=\frac{N^2}{n^2}$  задач, кожна з яких буде мати складність $\mathcal{O}(n^2)$ . Припустимо, що планувальник забезпечує пересилку повідомлень на вузли за певним фіксованим алгоритмом, який завершує обчислення за час $T(N,n)$. Тоді задача користувача полягає у пошуку мінімуму функції:

\begin{equation}
	\label{eq:general_minimization_problem}
	T(N,n) \longrightarrow \min
\end{equation}

Функція $T(N,n)$ , взагалі кажучи, може мати багато локальних мінімумів (в залежності від обчислювальної системи та планувальника) оскільки існує мінімальна фіксована величина задачі. Ілюстрації графіків функції, отриманої шляхом симуляції, наведені у експериментальній частині та корелють з отриманими результатами у \cite{DoroshenkoIgnatenkoIvanenko}.

Одним з розповсюджених підходів до аналізу таких задач полягає у дослідженні потокової моделі даного процесу \cite{FluidModelForJobScheduling}.

Припустимо, що користувач вибрав вектор $x \in \mathbb{R}$ з компонентами $x_i$, де $x_i > 0$ , $x_i \le k$, $i=1,\ldots,m$, $\sum_{i=1}^{k}x_i = k$. Всі таки вектори утворюють множину $X(n)$ . Кожен компонент вектора $x$ описує відсоток задач, призначених для виконання на  $i$-тому процесорі. Будемо брати до уваги тільки операції, множення. Таке спрощення дозволяє у явному вигляді виписати функції часу. Загальний час закінчення залежить від  $x$, та дорівнює:

\begin{equation}
	\label{eq:total_time_general}
	T(x,X(n)) = \max\limits_{i=1,\ldots,m}{\frac{x_i N n^2}{p_i}}
\end{equation}

Потокова модель передбачає можливість розділення задачі на підзадачі розміру $\epsilon = N n^2$, компонування з них підходящих підзадач та визначення загального часу при  $\epsilon \longrightarrow 0$.

Твердження 1.
Мінімальний час закінчення обчислень (без пересилок) для потокової моделі з одним користувачем дорівнює  . Відповідно, користувач, для досягнення оптимального часу, має розділити свою задачу таким чином, щоб  вузол   отримав   обчислень. 
Функція Мінковського для множини   та вектора   визначається наступним чином:
.
Відомо, що ця функція опукла для опуклої  .
Визначимо множину потужностей системи   та масштабуємо її наступним чином:
.
Тоді  . 
Доведення. Розглянемо праву частину: . Умова належності вектора   множині   записується у вигляді  , отже  .  У іншому вигляді  . З властивостей функції   випливає, що мінімальний час   існує і єдиний. Для врахування пересилок потрібно зазначити, що алгоритм надсилає   елементів на відповідний вузол та приймає   елементів. Отже, сумарний час закінчення з урахуванням пересилок дорівнює 
.
Твердження 2. [6] Існує мінімум часу по   -  . 

